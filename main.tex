%%%%%%%%%%%%%%%%% - PREÁMBULO - %%%%%%%%%%%%%%%%% 

% ----------------- Documento ----------------- %
% Se define el tipo de documento (en este caso un artículo), en hoja A4 con tamaño de fuente de 11pt, escrito en castellano, e indicando que el documento tendrá páginas distintas a izquierda y derecha ("twoside"):
\documentclass[a4paper, 11pt, spanish, twoside]{article}
% Los demás tipos de documentos así como sus características y opciones pueden consultarse en: https://en.wikibooks.org/wiki/LaTeX/Document_Structure#Document_classes
% ---------------------------------------------- %


% ------------------ Página -------------------- %
% Se define el tamaño de las páginas, indicando el tamaño de los márgenes superior e inferior ("top" y "bottom"), e izquierdo y derecho ("left" y "right"):
\usepackage[top=2.5cm,bottom=2.5cm,left=2.5cm,right=2.5cm]{geometry}
% Se inserta el comando \raggedbottom para evitar que LaTeX rellene con espacios en blanco aquellas páginas que no alberguen suficiente contenido como para rellenarlas de forma "natural":
\raggedbottom 
% ---------------------------------------------- %


% ------------- Paquetes generales ------------- %
% Se importan distintos paquetes de propósito general:
\usepackage[utf8]{inputenc}
\usepackage[spanish,es-tabla]{babel}
\usepackage{float}
\usepackage{caption}
% ---------------------------------------------- %


% ------------ Paquetes específicos ------------ %
% Se importan distintos paquetes que será utilizados en momentos concretos del documento: 
\usepackage{pdfpages} % Para insertar la portada en formato PDF.
\usepackage{amssymb} % Para símbolos matemáticos.
\usepackage{bm} % Para negrita en símbolos matemáticos.
\usepackage{amsmath} % Para el entorno "split".
\usepackage[hidelinks]{hyperref} % Para urls.
\usepackage{longtable} % Para tablas largas.
\usepackage{pgf} % Para gráficas en formato PGF.
\usepackage{graphicx} % Para insertar imágenes.
\usepackage{wrapfig} % Para posicionar imágenes alrededor del texto.
\usepackage{fontawesome5} % Para utilizar iconos de "fontawesome".
\usepackage{pdflscape}  % Para colocar páginas en formato apaisado.
\usepackage{array} % Para centrar verticalmente el contenido de las celdas de una tabla.
\usepackage{svg} % Add this line in the preamble to use the svg package
\usepackage{booktabs}
\usepackage{siunitx}
% ---------------------------------------------- %


% ---------------- Numeración ------------------ %
\counterwithin{table}{section} % Se numeran las tablas con respecto al capítulo en el que se encuentran.
\counterwithin{figure}{section} % Se numeran las figuras con respecto al capítulo en el que se encuentran.
\counterwithin{equation}{section} % Se numeran las ecuaciones con respecto al capítulo en el que se encuentran.
% ---------------------------------------------- %


% ------------- Página en blanco ----------------%
% Se define un comando (\blankpage) para insertar una página totalmente en blanco (sin número de página, encabezado y pie de página):
\usepackage{afterpage}
\newcommand\blankpage{%
    \null
    \thispagestyle{empty}%
    \newpage}
% ---------------------------------------------- %    


% ----------- Formato de los párrafos -----------%
% Se define el formato de los párrafos:
\setlength{\parindent}{0pt} % Se elimina la sangría en comienzo de párrafo (0pt).
\setlength{\parskip}{1em} % Se define el espacio entre dos párrafos (1em).
% ---------------------------------------------- %    


% -------------- Título adicional -------------- %
% Se añade una profundidad adicional a los títulos (profundidad 4):
\usepackage{titlesec}
\setcounter{secnumdepth}{4} % Se fija en 4 la profundidad de numeración de títulos.
\setcounter{tocdepth}{4} % Se fija en 4 la profundidad de títulos incluidos en el índice.
% Se modifica el formato de \paragraph (título de profundidad 4) para adaptarlo al formato del resto de títulos:
\titleformat{\paragraph}
{\normalfont\normalsize\bfseries}{\theparagraph}{1em}{}
\titlespacing*{\paragraph}
{0pt}{3.25ex plus 1ex minus .2ex}{1.5ex plus .2ex} 
% ---------------------------------------------- %    


% --------- Encabezado y pie de página -------- %
% El encabezado y pie de página forman parte del paquete fancyhdr:
\usepackage{fancyhdr}
\fancyhf{}
\pagestyle{fancy}

% Se ajusta el tamaño de fuente para el encabezado y pie de página (9pt)
\fancyhf{\fontsize{2}{12}\selectfont}

% Contenido del encabezado (\fancyhead):
\fancyhead[RO]{Título del trabajo} % Texto que se coloca en el encabezado de las páginas impares (O -> 'Odd', o impar) a la izquierda (R -> 'Odd')
\fancyhead[LE]{\nouppercase{\rightmark}} % Texto que se coloca en el encabezado de las páginas pares (E -> 'Even', o par) a la izquierda (L -> 'Left'). \rightmark se utiliza para insertar automáticamente el título de la sección correspondiente, y \nouppercase para que no aparezca todo en mayúsculas (formato por defecto de \rightmark).

% Contenido del pie de página (\fancyfoot):
\fancyfoot[RE]{Escuela  Técnica  Superior  de  Ingenieros  Industriales  (UPM)} % Texto que se coloca en el pie de página de las páginas pares (E -> 'Even', o par) a la derecha (R -> 'Right')
\fancyfoot[LO]{Gonzalo Aris Jiménez} % Texto que se coloca en el pie de página de las páginas impares (O -> 'Odd', o impar) a la izquierda (L -> 'Left')
\fancyfoot[LE,RO]{\thepage} % El número de página (\thepage) se coloca a la izquierda en las páginas pares y a la derecha en las impares.

% Se indica que sólo se quiere incorporar en \rightmark (utilizado más arriba) el título de la sección (y no de las subsecciones, subsubsecciones, etc.):
\renewcommand{\sectionmark}[1]{\markright{\thesection. #1}}
\renewcommand{\subsectionmark}[1]{}

% Formato de la línea de separación horizontal:
\renewcommand{\headrulewidth}{0.5pt} % Ancho de la línea del encabezado.
\renewcommand{\footrulewidth}{0.5pt} % Ancho de la línea del pie de página.
% ---------------------------------------------- % 

  
% -------------- Diagramas TikZ ---------------- %
% Se importa el paquete TikZ para diagramas de propósito general (diagramas de bloques, de flujo, etc.):
\usepackage{tikz}
\usetikzlibrary{shadows, trees, shapes.geometric}
\tikzset{
every picture/.append style={
  execute at begin picture={\deactivatequoting},
  execute at end picture={\activatequoting}
  }
}
% Se importa el paquete circuitikz para circuitos eléctricos:
\usepackage{circuitikz}
% ---------------------------------------------- % 


% ----------- Gráficas con Geogebra ------------ %
% Se importan los distintos paquetes para incorporar diagramas TikZ mediante Geogebra (paquetes adicionales a los importados en el bloque anterior, diagramas TikZ):
\usepackage{pgfplots}
\pgfplotsset{compat=1.15}
\usepackage{mathrsfs}
\usetikzlibrary{arrows, babel}
% ---------------------------------------------- % 


% ------------- Diagramas de Gantt ------------- %
% Se importan distintos paquetes y se realizan diferentes ajustes para representar insertar diagramas de Gantt en el documento:
\usepackage{pgfgantt}
\usepackage{geometry} 
\usepackage{pdflscape} 
\usepackage{ragged2e}
\usepackage{translator}
\ganttset{calendar week text={\small{\startday/\startmonth}}}
\newcommand\textganttbar[5][]{%
    \ganttbar[#1,bar/.append style={alias=tmp}]{#2}{#4}{#5}
    \path 
    let
    \p1=(tmp.west),\p2=(tmp.east),
    \n1={\x2-\x1},\n2={width("#3")},
    \n3={ifthenelse(\n1>\n2,90,270)}
    in
    node [anchor=\n3,font=\footnotesize] at (tmp.north) {#3};
}
% ---------------------------------------------- % 


% ----------- Fragmentos de código ------------- %
% El paquete utilizado para insertar fragmentos de código en el documento es listings. En el presente bloque del preámbulo se definen ciertos parámetros de listings con el objetivo de adaptar dicho paquete a código escrito en Python.

\usepackage{listings} % Paquete para insertar código. 
\usepackage{xcolor} % Paquete para definir colores.

% Se definen los distintos colores que se utilizan para resaltar ciertos elementos del código:
\definecolor{codegreen}{rgb}{0.04314,0.6745,0.07843} % Verde.
\definecolor{codegray}{rgb}{0.5,0.5,0.5} % Gris.
\definecolor{codered}{rgb}{0.5373,0.02745,0.06275} % Rojo.
\definecolor{codeblue}{rgb}{0.071,0.0258,0.9882} % Azul.
\definecolor{codepurple}{rgb}{0.6,0.02745,0.5961} % Morado.

% Se define el color de fondo:
\definecolor{backcolour}{rgb}{0.95,0.95,0.92} % Gris oscuro.

% Se define el valor de ciertos parámetros de listings para adaptar dicho paquete a código escrito en Python:
\lstdefinestyle{mystyle}{
    language=R,
    basicstyle=\ttfamily\small,
    backgroundcolor=\color{backcolour},
    commentstyle=\color{codegreen},
    keywordstyle=\color{codeblue},
    keywordstyle={[2]\color{codeblue}},
    keywordstyle={[3]\color{codepurple}},
    keywordstyle={[4]\bfseries},
    stringstyle=\color{codegreen},
    emphstyle=\color{codered},
    numbers=left,
    numberstyle=\tiny\color{codegray},
    numbersep=5pt,
    showspaces=false,
    showstringspaces=false,
    breaklines=true,
    postbreak=\mbox{{$\hookrightarrow$}\space},
    breakatwhitespace=true,
    tabsize=2,
    captionpos=b,
    frame=single,
    framesep=5pt,
    framerule=0.5pt,
    framexleftmargin=10pt, % Adjust the left margin
    framexrightmargin=5pt, % Adjust the right margin
    rulecolor=\color{codegray},
    xleftmargin=15pt,
    xrightmargin=15pt,
    aboveskip=10pt,
    belowskip=10pt,
    moredelim=[s][\color{codepurple}]{'}{'},
    moredelim=[s][\color{codepurple}]{`}{`},
    moredelim=[s][\color{codepurple}]{[}{]},
    moredelim=[s][\color{codepurple}]{[[}{]]},
    moredelim=[s][\color{codepurple}]{\$}{\ },
    moredelim=[l][\color{codegray}]{\#},
}

\lstset{style=mystyle} % Se asocia el estilo de listings al estilo que acaba de definirse ("mystyle")

% Se realizan una serie de operaciones complementarias con el paquete listings (su comprensión no es necesaria para manejar dicho paquete):
\makeatletter
\def\lst@OpLiteratekey#1\@nil@{\let\lst@ifxopliterate\lst@if
                             \def\lst@opliterate{#1}}
\lst@Key{opliterate}{}{\@ifstar{\lst@true \lst@OpLiteratekey}
                             {\lst@false\lst@OpLiteratekey}#1\@nil@}
\lst@AddToHook{SelectCharTable}
    {\ifx\lst@opliterate\@empty\else
         \expandafter\lst@OpLiterate\lst@opliterate{}\relax\z@
     \fi}
\def\lst@OpLiterate#1#2#3{%
    \ifx\relax#2\@empty\else
        \lst@CArgX #1\relax\lst@CDef
            {}
            {\let\lst@next\@empty
             \lst@ifxopliterate
                \lst@ifmode \let\lst@next\lst@CArgEmpty \fi
             \fi
             \ifx\lst@next\@empty
                 \ifx\lst@OutputBox\@gobble\else
                   \lst@XPrintToken \let\lst@scanmode\lst@scan@m
                   \lst@token{#2}\lst@length#3\relax
                   \lst@XPrintToken
                 \fi
                 \let\lst@next\lst@CArgEmptyGobble
             \fi
             \lst@next}%
            \@empty
        \expandafter\lst@OpLiterate
    \fi}

\lstset{ 
    literate={á}{{\'a}}1 {é}{{\'e}}1 {í}{{\'i}}1 {ó}{{\'o}}1 {ú}{{\'u}}1
  {Á}{{\'A}}1 {É}{{\'E}}1 {Í}{{\'I}}1 {Ó}{{\'O}}1 {Ú}{{\'U}}1
  {à}{{\`a}}1 {è}{{\`e}}1 {ì}{{\`i}}1 {ò}{{\`o}}1 {ù}{{\`u}}1
  {À}{{\`A}}1 {È}{{\'E}}1 {Ì}{{\`I}}1 {Ò}{{\`O}}1 {Ù}{{\`U}}1
  {ä}{{\"a}}1 {ë}{{\"e}}1 {ï}{{\"i}}1 {ö}{{\"o}}1 {ü}{{\"u}}1
  {Ä}{{\"A}}1 {Ë}{{\"E}}1 {Ï}{{\"I}}1 {Ö}{{\"O}}1 {Ü}{{\"U}}1
  {â}{{\^a}}1 {ê}{{\^e}}1 {î}{{\^i}}1 {ô}{{\^o}}1 {û}{{\^u}}1
  {Â}{{\^A}}1 {Ê}{{\^E}}1 {Î}{{\^I}}1 {Ô}{{\^O}}1 {Û}{{\^U}}1
  {Ã}{{\~A}}1 {ã}{{\~a}}1 {Õ}{{\~O}}1 {õ}{{\~o}}1
  {œ}{{\oe}}1 {Œ}{{\OE}}1 {æ}{{\ae}}1 {Æ}{{\AE}}1 {ß}{{\ss}}1
  {ű}{{\H{u}}}1 {Ű}{{\H{U}}}1 {ő}{{\H{o}}}1 {Ő}{{\H{O}}}1
  {ç}{{\c c}}1 {Ç}{{\c C}}1 {ø}{{\o}}1 {å}{{\r a}}1 {Å}{{\r A}}1
  {€}{{\euro}}1 {£}{{\pounds}}1 {«}{{\guillemotleft}}1
  {»}{{\guillemotright}}1 {ñ}{{\~n}}1 {Ñ}{{\~N}}1 {¿}{{?`}}1
  {º}{{\textordmasculine}}1}

\lstset{opliterate=
   *{0}{{{\color{codered}0}}}1 {1}{{{\color{codered}1}}}1 
   {2}{{{\color{codered}2}}}1 {3}{{{\color{codered}3}}}1 
   {4}{{{\color{codered}4}}}1 {5}{{{\color{codered}5}}}1 
   {6}{{{\color{codered}6}}}1 {7}{{{\color{codered}7}}}1 
   {8}{{{\color{codered}8}}}1 {9}{{{\color{codered}9}}}1}

\DeclareCaptionType{code}[Código][ÍNDICE DE CÓDIGOS] % Se define el entorno "Código" (de forma que al introducir un fragmento de código en el documento aparezca como: Código 1.1: ...), y la lista con los distintos códigos ("Índice de códigos").
\counterwithin{code}{section} % Se numeran los códigos con respecto al capítulo en el que se encuentran.
% ---------------------------------------------- % 


% --------------- Bibliografía ----------------- %
% El manejo de la bibliografía se realiza mediante el paquete biblatex:
\usepackage[backend=biber, style=authoryear, sorting=nyt, citestyle=authoryear, maxcitenames=2, maxbibnames=5, giveninits=true, uniquename=init]{biblatex} 

% Los distintos parámetros que aparecen en la línea anterior corresponden a las siguientes características de la bibliografía:
% - style: la manera en la que aparecen las referencias en la bibliografía. En este caso se opta por "authoryear", pero existen múltiples estilos posibles que se resumen en la siguiente guía: https://www.overleaf.com/learn/latex/biblatex_bibliography_styles.
% - sorting: orden en el que aparecen las distintas referencias en la bibliografía. En este caso se opta por ordenarlas en primer lugar por el apellido del primer autor, en segundo lugar por el año de publicación, y por último por el título de la publicación (nyt=name-year-title)
% - citestyle: elementos y orden de dichos elementos de una referencia al citarla en el documento. En este caso se escoge "authoryear" para que aparezca en primer lugar el apellido del autor (o de los autores) y en segundo lugar el año de publicación. Existe gran variedad de opciones en cuanto al parámetro citestyle que se resumen en: https://www.overleaf.com/learn/latex/biblatex_citation_styles.
% maxcitenames: máximo número de autores que aparecen al citar una referencia en el documento. Al escoger un valor de 2 para este parámetro se pueden dar los siguientes casos: un único autor -> (autor, año), dos autores -> (autor 1 y/e autor 2, año), tres o más autores -> (autor 1 et al., año).
% maxbibnames: parámetro idéntico al anterior pero para la bibliografía en lugar de las citas.
% giveinits y uniquename: para mostrar únicamente las iniciales de los nombres de los autores.

% Se importa el paquete csquotes para citar las referencias a lo largo del documento:
\usepackage{csquotes} 

% Se realizan una serie de operaciones para adaptar la bibliografía al estilo deseado (coma entre autor y año al citar una referencia, idioma castellano, etc.):
\DeclareNameAlias{sortname}{family-given}
\renewcommand*{\nameyeardelim}{\addcomma\space}
\setlength\bibitemsep{\baselineskip}
\DefineBibliographyStrings{spanish}{%
  andothers = {et\addabbrvspace al\adddot}
}

\makeatletter

\newrobustcmd*{\parentexttrack}[1]{%
  \begingroup
  \blx@blxinit
  \blx@setsfcodes
  \blx@bibopenparen#1\blx@bibcloseparen
  \endgroup}

\AtEveryCite{%
  \let\parentext=\parentexttrack%
  \let\bibopenparen=\bibopenbracket%
  \let\bibcloseparen=\bibclosebracket}

\makeatother

\addbibresource{biblio.bib}
% ---------------------------------------------- % 


%%%%%%%%%%%% - INICIO DEL DOCUMENTO - %%%%%%%%%%%%

\begin{document} 

%%%%%%%%%%%%%%%%%%%%%%%%%%%%%%%%%%%%%%%%%%%%%%%%%%


%%%%%%%%%%%%%%%%%%% - PORTADA - %%%%%%%%%%%%%%%%%%

% Se comienza una página nueva sin formato (sin número de página y sin encabezado/pie de página), ya que sólo incorpora la la portada:
\newpage
\thispagestyle{empty}

% La portada se inserta mediante el comando \includepdf seguido del archivo PDF correspondiente (que se ajusta automáticamente a las dimensiones de la página):
\includepdf{Portada_TFG.pdf}

%%%%%%%%%%%%%%%%%%%%%%%%%%%%%%%%%%%%%%%%%%%%%%%%%%

% Las páginas anteriores al contenido del TFG/TFM (previas a la introducción) suelen numerarse de forma distinta a las del cuerpo del informe, en este caso en números romanos:
\pagenumbering{roman}

%%%%%%%%%%%%%%%%%%%%%%%%%%%%%%%%%%%%%%%%%%%%%%%%%%


%%%%%%%%%%%%%$%%%%% - CITA - %%%%%%%%%%%%%%%%%%%%%
 
% Se comienza una página nueva sin formato (sin número de página y sin encabezado/pie de página), ya que sólo incorpora la cita:
\newpage
\thispagestyle{empty}

\begin{flushright} % Se alinea el texto en el lado derecho de la página.
\vspace*{5cm} % Se añade un espacio vertical de 5cm para situar la cita en ~1/3 de la página.

\textit{“La cita del trabajo iría aquí”} 

\medskip % Salto a la línea de tamaño medio (existen \smallskip, \medskip y \bigskip)
- El autor de la cita 

\end{flushright}

\afterpage{\blankpage} % Se añade una página en blanco después de la cita.

%%%%%%%%%%%%%%%%%%%%%%%%%%%%%%%%%%%%%%%%%%%%%%%%%%


%%%%%%%%%%%%% - AGRADECIMIENTOS - %%%%%%%%%%%%%%%%

% Se comienza una página nueva con formato plano (sin encabezado/pie de página pero con número de página):
\newpage
\thispagestyle{plain}

\section*{AGRADECIMIENTOS} % Se añade un asterisco a \section para que el título no esté numerado.
\addcontentsline{toc}{section}{AGRADECIMIENTOS} % Al utilizar \section* se ha de añadir manualmente el apartado al índice (Table Of Contents, TOC).

Agradezco a mis tutores Carolina y Andrés por su apoyo incondicional. Por todas la reuniones y dudas y por ser tan importantes en mi motivación por el mundo de la investigación.  

Doy gracias a mi novia Mariona, por estar a mi lado en todo momento y por ser mi apoyo incondicional en la elaboración de este trabajo, por su inteligencia y por acompañarme y motivarme en todo momento. 

Gracias a mis compañeros Pablo y Ana de la universidad de Aalto por su apoyo y compañía en la elaboración de este trabajo. 

Gracias a Jose Manuel y Alberto por invitarme a la Universidad de Alcalá, por a presentarme a su equipo y darme la oportunidad de presentar mi trabajo frente a ellos.

Gracias a Pablo Garrido, por ser un educador referente en la Universidad Politécnica de Madrid, por motivarnos y enseñarnos todos los días qué significa ser ingeniero, por hacernos ver que con una ingeniería se puede y por ser un ejemplo a seguir.

Gracias en último lugar a mi familia por ser familia y estar ahí siempre.

\newpage
%%%%%%%%%%%%%%%%%%%%%%%%%%%%%%%%%%%%%%%%%%%%%%%%%%


%%%%%%%%%%%%%% - RESUMEN EJECUTIVO - %%%%%%%%%%%%%

\section*{RESUMEN EJECUTIVO} % Se añade un asterisco a \section para que el título no esté numerado.
\markright{RESUMEN EJECUTIVO} % Al utilizar \section* se ha de añadir manualmente el título del apartado al encabezado.
\addcontentsline{toc}{section}{RESUMEN EJECUTIVO} % Al utilizar \section* se ha de añadir manualmente el apartado al índice (Table Of Contents, TOC).

\afterpage{\blankpage} % Se añade una página en blanco después del resumen.

%%%%%%%%%%%%%%%%%%%%%%%%%%%%%%%%%%%%%%%%%%%%%%%%%%


%%%%%%%%%%%%%%%%%%% - ÍNDICE - %%%%%%%%%%%%%%%%%%%

\newpage

\renewcommand*\contentsname{ÍNDICE} % Se modifica el nombre por defecto de la "Table Of Contents" (tabla de contenidos, índice) para pasar a llamarla "ÍNDICE".

\tableofcontents % Se genera el índice de contenidos del documento que incorpora todos los títulos de \section, \subsection y \subsubsection (y también \paragraph, ver capítulo 1), así como los títulos añadidos con \addcontentsline (como el resumen ejecutivo, por ejemplo).

\afterpage{\blankpage} % Se añade una página en blanco después del índice.

%%%%%%%%%%%%%%%%%%%%%%%%%%%%%%%%%%%%%%%%%%%%%%%%%%


%%%%%%%%%%%%%% - ÍNDICE DE TABLAS - %%%%%%%%%%%%%%

\newpage

\renewcommand{\listtablename}{ÍNDICE DE TABLAS} % Se define el nombre del índice de tablas.
\listoftables % Se genera automáticamente el índice con las distintas tablas del documento (entorno \table o \longtable).
\addcontentsline{toc}{section}{ÍNDICE DE TABLAS} % Se añade manualmente el apartado al índice (Table Of Contents, TOC).

%%%%%%%%%%%%%%%%%%%%%%%%%%%%%%%%%%%%%%%%%%%%%%%%%%


%%%%%%%%%%%%% - ÍNDICE DE FIGURAS - %%%%%%%%%%%%%%

\newpage

\renewcommand{\listfigurename}{ÍNDICE DE FIGURAS} % Se define el nombre del índice de figuras.
\listoffigures % Se genera automáticamente el índice con las distintas figuras del documento (entorno \figure).
\addcontentsline{toc}{section}{ÍNDICE DE FIGURAS} % Se añade manualmente el apartado al índice (Table Of Contents, TOC).

%%%%%%%%%%%%%%%%%%%%%%%%%%%%%%%%%%%%%%%%%%%%%%%%%%


%%%%%%%%%%%%%% - ÍNDICE DE CÓDIGOS - %%%%%%%%%%%%%

\newpage

\listofcodes % Se genera automáticamente el índice con los distintos códigos del documento (entorno \code).
\addcontentsline{toc}{section}{ÍNDICE DE CÓDIGOS} % Se añade manualmente el apartado al índice (Table Of Contents, TOC).

\afterpage{\blankpage} % Se añade una página en blanco después del índice de códigos.

%%%%%%%%%%%%%%%%%%%%%%%%%%%%%%%%%%%%%%%%%%%%%%%%%%


%%%%%%%%%%%%%%%%%%%%%%%%%%%%%%%%%%%%%%%%%%%%%%%%%%

% Se inicia una nueva página, y se restablece la numeración de las páginas, utilizando esta vez el sistema de numeración estándar (1, 2, 3, 4, ...)
\newpage
\pagenumbering{arabic}

%%%%%%%%%%%%%%%%%%%%%%%%%%%%%%%%%%%%%%%%%%%%%%%%%%

\section{INTRODUCTION} \label{sec:itroduction}

\subsection{Motivation for the research}

\subsection{Outline of the thesis}

\newpage

\section{OBJECTIVES} \label{sec:objectives}

\subsection{Research objectives}

\subsection{Research questions}


\newpage

\section{METODOLOGÍA}\label{sec:methodology}

En este capítulo, se describirá la metodología utilizada para conseguir cumplir los objetivos establecidos. 

Se comenzará por discutir el diseño de la investigación y los métodos de recolección de datos que han sido utilizados para recopilar información. Luego se describirá el enfoque aportado para el preprocesamiento de datos, incluida la limpieza de datos y la selección de características.

Finalmente, se describirán los algoritmos de aprendizaje automático utilizados para el estudio y clasificación de pacientes. 

\subsection{Diseño de Investigación}

Se ha utilizado un enfoque de investigación cuantitativa para examinar la progresión temporal de la bronquiolitis en pacientes pediátricos. Se ha partido de 47 variables descriptivas y de 2 variables que muestran la evolución temporal de los pacientes durante las primeras 24 h de ingreso y que han sido tratadas en el presente trabajo como series temporales. 

\subsubsection{Tipos de Variables}\label{sec:tiposdevariables}

Dentro de las 47 variables descriptivas a utilizar antes mencionadas también se incluyen 4 variables que muestran la evolución temporal por intervalos de los pacientes. Estas variables son:

\begin{itemize}
    \item Flujo de Oxígeno
    \item Frecuencia Respiratoria
    \item Escala SAPI (Sistemas de Alerta Precoz Infantil)
    \item Score Wood-Downes
\end{itemize}

Estas 4 variables no han sido tratadas como series temporales dada la baja frecuencia de recolección durante las primeras 24 h de ingreso; que es el intervalo temporal al que ha sido acotado el estudio. Las tres primeras (Flujo de Oxígeno, Frecuencia Respiratoria y Escala SAPI)variables han sido recopiladas 3 veces durante las primeras 24h del ingreso del paciente pediátrico y la última (Score Wood-Downes) ha sido recogida a la llegada del paciente y a las 24 h. Es decir estas variables describen la estancia del paciente en intervalos.  Estas variables serán catalogadas como \textit{Temporales en Intervalos.} 

Si tratamos estas 3 últimas variables como temporales nos quedarían solamente 36 variables descriptivas. (3 variables temporales $\times$ 3 intervalos + 1 $\times$ 2 intervalos = 11 variables) y por otro lado 2 variables en forma de series temporales. Cada variable de estas contiene 1440 datos. (60 minutos $\times$ 24 horas = 1440).

Dentro de las 36 variables descriptivas restantes se encuentran 3 variables que dan información más allá de las primeras 24 h de monitorización. Estas variables en principio serán excluidas del estudio y son:

\begin{itemize}
    \item Días con Gafas Nasales
    \item Días con O$_2$
    \item Días con OAF
\end{itemize}

Estas 3 variables serán catalogadas como: \textit{Descriptivas fuera del scope}. El \textit{scope} será básicamente las primeras 24 h de ingreso del paciente pediátrico.

Por último dentro de las 33 variables descriptivas dentro del \textit{scope} se encuentran 2 variables que no son ni cualitativas ni cuantitativas. Estas variables serán catalogadas como \textit{Otras}. Las 31 variables restantes serán catalogadas como \textit{Descriptivas dentro de scope}.

En la tabla \ref{tabla:variables_estudio} se muestran las diferentes variables recopiladas para realizar el presente estudio. 

\begin{table}[H]
    \centering
        \begin{tabular}{| m{5cm} | m{1.75cm} | m{7cm} |}
            \hline Tipo de Variable & Cantidad & Nombres  \\ \hline
            Descriptivas dentro de scope & 31 & Edad, Peso, Sexo, Edad Gestacional (EG), Palivizumab, Lactancia Materna (LM), Dermatitis, Alergias, Tabaco, Enfermedad Base, Radiografía, Analítica, Suero, Etiología, Prematuridad, Alimentación, Sonda Nasogástrica, Gafas Nasales al Ingreso, OAF, OAF al ingreso, OAF tras ingreso, Horas de Ingreso tras inicio OAF, UCIP, Deterioro, Pausas de Apnea, PCT (Procalcitonina en la sangre), PCR (Prueba de Proteína C relativa), Leucocitos, Nautrófilos, Linfocitos y Score Cruces Ingreso  \\ \hline
            Descriptivas fuera de scope & 3 & Días con Gafas Nasales, Días con O$_2$ y Días con OAF. \\ \hline
            Temporales en 3 Intervalos & 11 & Frecuencia Respiratoria (0 - 8 h), Frecuencia Respiratoria (8 - 16 h),
            Frecuencia Respiratoria (16 - 24 h),
            Flujo O$_2$ (0 - 8 h),
            Flujo O$_2$ (8 - 16 h),
            Flujo O$_2$ (16 - 24 h),
            SAPI (0 - 8 h),
            SAPI (8 - 16 h), 
            SAPI (16 - 24 h), Score Wood-Downes Ingreso y Score Wood-Downes 24 h . \\ \hline
            Series Temporales & 2 & Frecuencia Cardiaca, Saturación de Oxígeno \\ \hline
            Otras & 2 & Notas e Identificador Paciente. \\ \hline
        \end{tabular}
    \caption{Variables Usadas en el Estudio}\label{tabla:variables_estudio}
\end{table}

Las variables temporales han sido recopiladas durante las primeras 24 h del ingreso del paciente pediátrico cuando mostraba un cuadro bronquiolítico. La frecuencia con la que han sido recopilados los datos ha sido de 1 vez cada minuto. En la siguiente figura \ref{fig:fc-JJB} se muestra un ejemplo de la evolución  variable \textit{Frecuencia Cardiaca} en forma de serie temporal y de la misma forma la figura \ref{satO2-JJB} pero para la \textit{Saturación de O$_2$}. Estas dos series temorales pertenecen a la evolución del paciente pediátrico \textit{JJB\_11182744}.

\begin{figure}[H]
    \centering
    \includegraphics[scale=0.70]{./img/Heart-Rate-JJB.png}
    \caption{Valores de Frecuencia Cardíaca del paciente \textit{JJB\_11182744}}
    \label{fig:fc-JJB}
\end{figure}

\begin{figure}[H]
    \centering
    \includegraphics[scale=0.70]{./img/SatO2-JJB.png}
    \caption{Valores de Saturación de O$_2$ del paciente \textit{JJB\_11182744}}
    \label{fig:satO2-JJB}
\end{figure}

A la hora de trabajar con variables temporales se ha de tener en cuenta que los datos temporales pueden ser de dos tipos: \textit{Discretos} o \textit{Continuos}. Los datos discretos son aquellos que se recogen en intervalos de tiempo, por ejemplo, el número de pacientes que llegan a un hospital cada hora. Los datos continuos son aquellos que se recogen de forma continua, por ejemplo, en nuestro caso la saturación y frecuencia cardíaca de un paciente cada minuto.

Para terminar este punto \ref{tiposdevariables} una última cuestión a valorar son las variables cualitativas y cuantitativas que se han recopilado. Las variables cualitativas son aquellas que describen una cualidad del paciente, por ejemplo, el sexo o la edad gestacional. Las variables cuantitativas son aquellas que describen una cantidad del paciente, por ejemplo, la frecuencia cardíaca o la saturación de oxígeno.

En la siguiente tabla se muestra la división entre variables cualitativas y cuantitivas recopiladas en el estudio y dentro del \textit{scope}.

\begin{table}[H]
    \centering
        \begin{tabular}{| m{5cm} | m{1.75cm} | m{7cm} |}
            \hline Tipo de Variable & Cantidad & Nombres  \\ \hline
            Cuantitativas & 15 & Edad, Peso, Edad Gestacional (EG), Horas de Ingreso tras inicio OAF, PCT (Procalcitonina en la sangre), PCR (Prueba de Proteína C relativa), Leucocitos, Nautrófilos, Linfocitos, Frecuencia Respiratoria (0 - 8 h), Frecuencia Respiratoria (8 - 16 h),
            Frecuencia Respiratoria (16 - 24 h),
            Flujo O$_2$ (0 - 8 h),
            Flujo O$_2$ (8 - 16 h),
            Flujo O$_2$ (16 - 24 h). \\ \hline
            Cualitativas & 27 & Sexo, Palivizumab, Lactancia Materna (LM), Dermatitis, Alergias, Tabaco, Enfermedad Base, Radiografía, Analítica, Suero, Etiología, Prematuridad, Alimentación, Sonda Nasogástrica, Gafas Nasales al Ingreso, OAF, OAF al ingreso, OAF tras ingreso, UCIP, Deterioro, Pausas de Apnea, Score Cruces Ingreso, SAPI (0 - 8 h),
            SAPI (8 - 16 h), 
            SAPI (16 - 24 h), Score Wood-Downes Ingreso y Score Wood-Downes 24 h. \\ \hline
            Otras & 2 & Notas e Identificador Paciente. \\ \hline
        \end{tabular}
    \caption{Variables Cualitativas y Cuantitativas Dentro del \textit{Scope}}\label{tabla:cuali_cuanti}
\end{table}

\subsection{Data Collection}

We collected the data for our study from a pediatric hospital, where we recruited infants diagnosed with heart disease. We obtained informed consent from the parents or guardians of the infants before recording the ECG data. The ECG recordings were taken over a period of 24 hours, and we collected a total of X hours of data.

\subsection{Data Preprocessing}

We preprocessed the ECG data using a custom pipeline, which included data cleaning and feature selection. We removed any noisy or missing data from the ECG recordings, and we extracted a set of relevant features that we believed were important for classification of heart rate variability.

\subsection{Machine Learning Algorithms}

We employed several machine learning algorithms to classify the heart rate variability in infants with heart disease. These included Random Forest, Support Vector Machines, and Long Short-Term Memory networks. We trained each model on a subset of the data, and we used cross-validation to assess their performance.

\subsection{Performance Metrics}

We evaluated the performance of each machine learning model using several metrics, including accuracy, precision, recall, and F1 score. We also used confusion matrices and Receiver Operating Characteristic (ROC) curves to assess the models' ability to correctly classify the ECG data.

By following this methodology, we were able to achieve our research objectives and gain insights into the heart rate variability in infants with heart disease. Our results and analysis are presented in Chapter 5.



\newpage

\section{METODOLOGÍA}\label{sec:methodology}

En este capítulo, se describirá la metodología utilizada para conseguir cumplir los objetivos establecidos. 

Se comenzará por discutir el diseño de la investigación y los tipos de variables que serán utilizados a la hora de realizar el estudio. Luego se describirá el enfoque aportado para el preprocesamiento de datos, incluida la limpieza de datos y la selección de características.

Finalmente, se describirán los algoritmos de aprendizaje automático utilizados para el estudio y clasificación de pacientes. 

\subsection{Diseño de Investigación}

Se ha utilizado un enfoque de investigación cuantitativa para examinar la progresión temporal de la bronquiolitis en pacientes pediátricos. Se ha partido de 47 variables descriptivas y de 2 variables que muestran la evolución temporal de los pacientes durante las primeras 24 h de ingreso y que han sido tratadas en el presente trabajo como series temporales. 

\newpage

\section{RESULTADOS}\label{cap:results}

\subsection{Estructura de los Resultados}\label{sec:results}

En este capítulo se presentan los resultados de la investigación mostrada en Metodología~\ref{cap:metodologia}. Se ha dividido en dos partes, la primera muestra los resultados del Estudio 1 y la segunda los resultados del Estudio 2.

\subsection{Resultados del Estudio 1}\label{sec:resultados-estudio-1}

\subsubsection{Frecuencia Cardiaca}

{\color{blue} Sobre el tema del acento: https://www.revespcardiol.org/es-cardiaco-o-cardiaco-articulo-S0300893217305031}

A continuación, se muestra el análisis de la varianza realizado para la \textit{Frecuencia Cardiaca}, la \textit{Frecuencia Cardiaca Escalada} y la \textit{Frecuencia Cardiaca Transformada por Cuantiles} entre pacientes que sufren DETERIORO y los que no. Se ha realizado el análisis de la varianza para cada una de las medias horarias. El resultado se muestra en la Tabla~\ref{tab:mean-FC}.  


\begin{table}[H]
    \centering
    \begin{tabular}{|c|r|r|r|}
        \hline
        \textbf{Media Horaria} & \textbf{p-valor} & \textbf{p-valor 
        cuantiles} & \textbf{p-valor 
        escalada} \\
        \hline
        MEAN\_1 & 0.129 & 0.142 & 0.902 \\
        MEAN\_2 & 0.254 & 0.412 & 0.825 \\
        MEAN\_3 & 0.143 & 0.105 & 0.925 \\
        MEAN\_4 & 0.331 & 0.309 & 0.867 \\
        MEAN\_5 & 0.077 & 0.042 & 0.355 \\
        MEAN\_6 & 0.098 & 0.1 & 0.5 \\
        MEAN\_7 & 0.357 & 0.232 & 0.474 \\
        MEAN\_8 & 0.182 & 0.218 & 0.975 \\
        MEAN\_9 & 0.158 & 0.241 & 0.83 \\
        MEAN\_10 & 0.316 & 0.423 & 0.211 \\
        MEAN\_11 & 0.609 & 0.637 & 0.045 \\
        MEAN\_12 & 0.035 & 0.005 & 0.382 \\
        MEAN\_13 & 0.293 & 0.144 & 0.603 \\
        MEAN\_14 & 0.169 & 0.17 & 0.727 \\
        MEAN\_15 & 0.214 & 0.335 & 0.722 \\
        MEAN\_16 & 0.094 & 0.136 & 0.652 \\
        MEAN\_17 & 0.127 & 0.132 & 0.943 \\
        MEAN\_18 & 0.476 & 0.816 & 0.56 \\
        MEAN\_19 & 0.128 & 0.086 & 0.731 \\
        MEAN\_20 & 0.2 & 0.186 & 0.947 \\
        MEAN\_21 & 0.391 & 0.592 & 0.643 \\
        MEAN\_22 & 0.157 & 0.198 & 0.875 \\
        MEAN\_23 & 0.1 & 0.026 & 0.445 \\
        MEAN\_24 & 0.204 & 0.187 & 0.95 \\
        \hline
    \end{tabular}
    \caption{p-valor de la media horaria de la \textit{Frecuencia Cardiaca} y la \textit{Frecuencia Cardiaca Transformada por Cuantiles} entre pacientes que sufren OAF y los que no}\label{tab:mean-FC}
\end{table}

De manera gráfica el resultado es el siguiente mostrado en la Figura~\ref{fig:mean-FC}. Se han añadido dos rectas verticales que muestran el nivel del valor $\alpha$ = 0.05 y $\alpha$ = 0.001. {\color{blue} ¿0.01?}Se puede observar que la mayoría de las medias horarias no son significativas, por lo que no se puede rechazar la hipótesis nula. Esto significa que no hay diferencias significativas entre las medias horarias de la \textit{Frecuencia Cardíaca} entre pacientes que sufren DETERIORO y los que no.

Se puede observar como de manera general que a los pacientes que se les suministra OAF tienen una mayor \textit{Frecuencia Cardíaca} que los que no. Esto se puede observar en la Figura~\ref{fig:fc-boxplot-mean}, aun así no se puede afirmar que esto tenga un efecto significativo de manera general en toda la monitorización del paciente.

Las medias más significativas son las referentes a las horas 12 y 23. Esta situación va a causar a la hora de realizar el \textit{Estudio 2} se tendrán problemas a la hora de realizar un modelo de clasificación dónde los valores de monitorización de la \textit{Frecuencia Cardíaca} sean significativos a la hora de clasificar pacientes como DETERIORO o no.

\begin{figure}[H]
    \centering
    \includegraphics[scale = 1]{./img/mean-FC.png}
    \caption{p-valor de la media horaria de la \textit{Frecuencia Cardiaca} entre pacientes que sufren DETERIORO y los que no}
    \label{fig:mean-FC}
\end{figure}

\newpage
\thispagestyle{empty}
% Se modifica la geometría (los márgenes) de la página y se coloca en formato horizontal:
\newgeometry{top=10mm, bottom=10mm, left=12mm, right=12mm}
\begin{landscape}
\begin{figure}[H]
    \centering
    \includegraphics[scale = 0.68]{./img/fc-boxplot-mean.png}
    \caption{Media Horaria de la \textit{Frecuencia Cardíaca}, \textit{Frecuencia Cardíaca Escalada} y la \textit{Frecuencia Cardíaca Transformada por Cuantiles} entre pacientes que sufren DETERIORO y los que no}
    \label{fig:fc-boxplot-mean}
\end{figure}
\end{landscape}
\restoregeometry

\subsubsection{Saturación de Oxígeno}

A continuación, se muestra el análisis de la varianza realizado para la \textit{Saturación de O$_2$} y la \textit{Saturación de O$_2$ Escalada} entre pacientes que sufren DETERIORO y los que no. Se ha realizado el análisis de la varianza para cada una de las medias horarias. El resultado se muestra en la Tabla~\ref{tab:mean-SatO2}.

\begin{table}[H]
    \centering
    \begin{tabular}{|c|r|r|}
        \hline
        \textbf{Media Horaria} & \textbf{p-valor} & \textbf{p-valor escalada} \\
        \hline
        MEAN\_1 & 0.151 & 0.176 \\
        MEAN\_2 & 0.015 & 0.017 \\
        MEAN\_3 & 0.52 & 0.778 \\
        MEAN\_4 & 0.343 & 0.608 \\
        MEAN\_5 & 0.349 & 0.718 \\
        MEAN\_6 & 0.672 & 0.639 \\
        MEAN\_7 & 0.141 & 0.078 \\
        MEAN\_8 & 0.868 & 0.769 \\
        MEAN\_9 & 0.571 & 0.939 \\
        MEAN\_10 & 0.549 & 0.842 \\
        MEAN\_11 & 0.991 & 0.643 \\
        MEAN\_12 & 0.44 & 0.581 \\
        MEAN\_13 & 0.312 & 0.427 \\
        MEAN\_14 & 0.968 & 0.797 \\
        MEAN\_15 & 0.641 & 0.966 \\
        MEAN\_16 & 0.799 & 0.183 \\
        MEAN\_17 & 0.312 & 0.993 \\
        MEAN\_18 & 0.89 & 0.204 \\
        MEAN\_19 & 0.543 & 0.71 \\
        MEAN\_20 & 0.384 & 0.784 \\
        MEAN\_21 & 0.867 & 0.503 \\
        MEAN\_22 & 0.665 & 0.195 \\
        MEAN\_23 & 0.882 & 0.294 \\
        MEAN\_24 & 0.889 & 0.358 \\
        \hline
    \end{tabular}
    \caption{p-valor de la media horaria de la \textit{Saturación de O$_2$} entre pacientes que sufren OAF y los que no}\label{tab:mean-SatO2}
\end{table}

Al igual que en el apartado anterior el resultado es el siguiente mostrado en la Figura~\ref{fig:mean-SatO2}. La mayoría de las medias horarias no son significativas, por lo que no se puede rechazar la hipótesis nula. Esto significa que no hay diferencias significativas entre las medias horarias de la \textit{Saturación de O$_2$} entre pacientes que sufren DETERIORO y los que no.

Se puede observar como de manera general que a los pacientes que se les suministra OAF tienen una mayor \textit{Saturación de O$_2$} que los que no. Esto se puede observar en la Figura~\ref{fig:satO2-boxplot-mean}, aun así no se puede afirmar que esto tenga un efecto significativo de manera general en toda la monitorización del paciente.

La única media significativa menor que $\alpha = 0.05$ es la monitorizada en la hora $2$. Al igual que en el apartado anterior, esta situación va a causar a la hora de realizar el \textit{Estudio 2} se tendrán problemas a la hora de realizar un modelo de clasificación dónde los valores de monitorización de la \textit{Saturación de O$_2$} sean significativos a la hora de clasificar pacientes como DETERIORO o no.

{\color{blue} Hay algunas cosas que pueden estar influyendo en estos resultados "negativos": (1) El supuesto de varianzas iguales; (2) El supuesto de distribuciones normales. Ambos supuestos son los que conducen a la distribución t con n+m-2 grados de libertad. Por último, (3) la presencia de atípicos.}

\begin{figure}[H]
    \centering
    \includegraphics[scale = 1]{./img/mean-SatO2.png}
    \caption{p-valor de la media horaria de la \textit{Saturación de Oxígeno} entre pacientes que sufren DETERIORO y los que no}
    \label{fig:mean-SatO2}
\end{figure}

\newpage
\thispagestyle{empty}
% Se modifica la geometría (los márgenes) de la página y se coloca en formato horizontal:
\newgeometry{top=10mm, bottom=10mm, left=12mm, right=12mm}
\begin{landscape}
\begin{figure}[H]
    \centering
    \includegraphics[scale = 0.68]{./img/satO2-boxplot-mean.png}
    \caption{Media Horaria de la \textit{Saturación de Oxígeno} y la \textit{Saturación de Oxígeno Escalada} entre pacientes que sufren DETERIORO y los que no}
    \label{fig:satO2-boxplot-mean}
\end{figure}
\end{landscape}
\restoregeometry


\subsection{Resultados: Estudio 2}\label{sec:resultados-estudio-1}

En esta sección de resultados, presentaremos los hallazgos obtenidos en el \textit{Estudio 2}. La sección se subdivide en cuatro subsecciones, y en cada una de ellas se mostrarán los siguientes resultados:

\begin{enumerate}
    \item Dendrograma de los clústeres obtenidos.
    \item Distribución de los clústeres obtenidos en función de las dos primeras componentes principales.
    \item Puntuación de Silhouette de los clústeres obtenidos.
    \item Dendrograma de los clústeres obtenidos según los pacientes que han experimentado OAF, junto con la tabla de contingencia.
    \item Clasificación mediante Random Forest de los clústeres según las variables \textit{Cuantitativas} y \textit{Cualitativas} de la Tabla~\ref{tabla:variables_estudio_final}, así como la Importancia de las cinco primeras variables.
    \item Clasificación Discriminante mediante Random Forest de los clústeres en función de los datos utilizados para generar los mismos clústeres (Raw Data, FAC, Peridiograma y FACC). Se mostrará la importancia para identificar qué datos contribuyen más en la clasificación entre los clústeres obtenidos.
    \item Cálculo de las medias de todos los valores empleados (Raw Data, FAC, Peridiograma y FACC) entre todos los pacientes, y diagrama de sus valores por clústeres.
\end{enumerate}

\paragraph{Leyenda de los gráficos}

\begin{enumerate}
    \item \textbf{Frecuencia Cardiaca}: \textit{HR}.
    \item \textbf{Saturación de Oxígeno}: \textit{SpO2}.
    \item \textbf{Frecuencia Cardíaca Escalada}: \textit{HR\_scaled}.
    \item \textbf{Saturación de Oxígeno Escalada}: \textit{SpO2\_scaled}.
    \item \textbf{Frecuencia Cardiaca Transformada por Cuantiles}: \textit{HR\_quantile}.
\end{enumerate}


\subsubsection{Raw Data}

\paragraph{Dendrograma de los clústeres obtenidos}

\begin{figure}[H]
    \centering
    
    \subfigure[\textit{HR}]{\includegraphics[width=0.45\textwidth]{img/01eucl-den.png}}
    \subfigure[\textit{HR\_scaled}]{\includegraphics[width=0.45\textwidth]{img/02eucl-den.png}}
    \subfigure[\textit{HR\_quantile}]{\includegraphics[width=0.5\textwidth]{img/03eucl-den.png}}
    \caption{Dendogramas de \textit{HR}, \textit{HR\_scaled} y \textit{HR\_quantile}}
    \label{fig:raw_data_den_fc}
\end{figure}

\begin{figure}[ht]
    \centering
    \subfigure[\textit{SpO2}]{\includegraphics[width=0.5\textwidth]{img/04eucl-den.png}}\hfill
    \subfigure[\textit{SpO2\_scaled}]{\includegraphics[width=0.5\textwidth]{img/05eucl-den.png}}
    \caption{Dendogramas de \textit{SpO2} y \textit{SpO2\_scaled}}\label{fig:raw_data_den_spo2}
\end{figure}

{\color{blue} Aunque el Silhouette sugiera dos clusters en la figura 5.5(a), es claro que el cluster azul es un dato atípico. Lo más sencillo es excluirlo del análisis y repetir el dendrograma.}

\paragraph{Distribución de los clústeres obtenidos en función de las dos primeras componentes principales}

\begin{figure}[H]
    \centering
    \subfigure[\textit{HR}]{\includegraphics[width=0.45\textwidth]{img/01eucl-pc.png}}
    \subfigure[\textit{HR\_scaled}]{\includegraphics[width=0.45\textwidth]{img/02eucl-pc.png}}
    \subfigure[\textit{HR\_quantile}]{\includegraphics[width=0.5\textwidth]{img/03eucl-pc.png}}
    \caption{Cluster Plot de \textit{HR}, \textit{HR\_scaled} y \textit{HR\_quantile}}
    \label{fig:raw_data_pc_fc}
\end{figure}

\begin{figure}[ht]
    \centering
    \subfigure[\textit{SpO2}]{\includegraphics[width=0.5\textwidth]{img/04eucl-pc.png}}\hfill
    \subfigure[\textit{SpO2\_scaled}]{\includegraphics[width=0.5\textwidth]{img/05eucl-pc.png}}
    \caption{Cluster Plot de \textit{SpO2} y \textit{SpO2\_scaled}}\label{fig:raw_data_pc_spo2}
\end{figure}


\paragraph{Puntuación de Silhouette de los clústeres obtenidos}

\begin{figure}[H]
    \centering
    \subfigure[\textit{HR}]{\includegraphics[width=0.45\textwidth]{img/01eucl-si.png}}
    \subfigure[\textit{HR\_scaled}]{\includegraphics[width=0.45\textwidth]{img/02eucl-si.png}}
    \subfigure[\textit{HR\_quantile}]{\includegraphics[width=0.5\textwidth]{img/03eucl-si.png}}
    \caption{Silhouette Plot de \textit{HR}, \textit{HR\_scaled} y \textit{HR\_quantile}}\label{fig:raw_data_si_fc}
\end{figure}

\begin{figure}[ht]
    \centering
    \subfigure[\textit{SpO2}]{\includegraphics[width=0.5\textwidth]{img/04eucl-si.png}}\hfill
    \subfigure[\textit{SpO2\_scaled}]{\includegraphics[width=0.5\textwidth]{img/05eucl-si.png}}
    \caption{Silhouette Plot de \textit{SpO2} y \textit{SpO2\_scaled}}\label{fig:raw_data_si_spo2}
\end{figure}


\subsubsection{FAC}

\subsubsection{Periodograma}

\subsubsection{FACC}
\newpage

\section{CONCLUSIONES Y LÍNEAS FUTURAS}\label{cap:conclusionesANDlineasfuturas}

\subsection{Conclusiones}\label{sec:conclusiones}


\newpage

\section{PLANIFICACIÓN Y PRESUPUESTO}\label{cap:results}

\subsection{Planificación}\label{sec:planificacion}

\subsubsection{Estructura de la Descomposición del Trabajo de Fin de Máster}\label{sec:edt}

La estructura de la descomposición del trabajo de fin de máster se muestra en la Figura \ref{fig:tfg_structure}. Se ha dividido en tres grandes bloques: \textit{Estudio Inicial}, \textit{Planteamiento y Modelado} y \textit{Redacción y Conclusiones}. 

El primer bloque se centra en la recopilación, limpieza e imputación de los datos. El segundo bloque se centra en el planteamiento de los objetivos, el modelado de los datos, la interpretación de los resultados y el contraste de los mismos. El tercer bloque se centra en la redacción del informe, la introducción, el desarrollo de la metodología, los resultados y el análisis, las conclusiones, las contribuciones y la revisión y edición del informe.

\thispagestyle{empty}
\newgeometry{top=10mm, bottom=10mm, left=12mm, right=12mm}
\begin{landscape}

\begin{figure}
\centering
\begin{tikzpicture}[scale=0.9,
every node/.style = {draw, rounded corners=3pt, semithick, drop shadow, font=\small},
ROOT/.style = {color=gray!70, text width = 14.5cm, inner sep=2mm, text=white, font=\bfseries, text centered},
L1/.style = {fill=white, text width = 3.5cm},
L2/.style = {fill=white, text width = 3cm, grow=down, xshift=1em, anchor=west, edge from parent path={(\tikzparentnode.south) |- (\tikzchildnode.west)}},
edge from parent/.style = {draw, thick},
LD/.style = {level distance=#1ex},
level 1/.style = {sibling distance=90mm}
]

\node[ROOT] {Trabajo de Fin de Máster}
[edge from parent fork down]
    child[LD=14] {node[L1, text width=6cm] {1. Estudio Inicial}
        child[L2,LD=9]   {node[L2]   {1.1 Recopilación de Datos}}
        child[L2,LD=18]  {node[L2]   {1.2 Limpieza de Datos}}
        child[L2,LD=28]   {node[L2]   {1.3 Imputación de Datos}
            child[L2,LD=11] {node[L2]   {1.3.1 Métodos de Imputación}}
            child[L2,LD=21] {node[L2]   {1.3.2 Evaluación de Imputaciones}}
            child[L2,LD=32] {node[L2]   {1.3.3 Selección de Imputaciones Finales}}
            }
        }
    child[LD=14]{node[L1, LD=40] {2. Planteamiento y Modelado}
        child[L2,LD=9]  {node[L2, text width=4.2cm]   {2.1 Planteamiento de Objetivos}
            child[L2,LD=9] {node[L2, text width = 3.5cm]   {2.1.1 Definición de Problema}}
            child[L2,LD=18] {node[L2, text width = 3.5cm]   {2.1.2 Determinación de Metas}}
            }
        child[L2,LD=37]  {node[L2, text width=4.3cm]   {2.2 Modelado de Datos}
            child[L2,LD=11] {node[L2, text width = 3.5cm]   {2.2.1 Selección de Modelos}}
            child[L2,LD=23] {node[L2, text width = 3.5cm]   {2.2.2 Entrenamiento de Modelos}}
            child[L2,LD=34] {node[L2, text width = 3.5cm]   {2.2.3 Evaluación de Modelos}}
            }
        child[L2,LD=81]  {node[L2, text width=5cm]   {2.3 Interpretación y Resultados}}
        child[L2,LD=89]  {node[L2, text width=5cm]   {2.4 Contraste de Resultados}}
        child[L2,LD=96]  {node[L2, text width=3.5cm]   {2.5 Limitaciones y Futuras Investigaciones}}
        }
    child [LD=14]{node[L1, LD=40] {3. Redacción y Conclusiones}
        child[L2,LD=9] {node[L2, text width=3.3cm]   {3.1 Estructura del Informe}}
        child[L2,LD=19] {node[L2, text width=3.3cm]   {3.2 Introducción}}
        child[L2,LD=29] {node[L2, text width=4.6cm]   {3.3 Desarrollo de la Metodología}}
        child[L2,LD=39] {node[L2, text width=4.6cm]   {3.4 Resultados y Análisis}}
        child[L2,LD=49] {node[L2, text width=4.6cm]   {3.5 Conclusiones}}
        child[L2,LD=59] {node[L2, text width=4.6cm]   {3.6 Contribuciones}}
        child[L2,LD=69] {node[L2]   {3.7 Revisión y Edición}}
    };
        
\end{tikzpicture}
\vspace{10pt}
\caption{Estructura para Trabajo de Fin de Máster (TFM)}
\label{fig:tfg_structure}
\end{figure}
\end{landscape}
\restoregeometry

\subsubsection{Diagrama de Gantt}\label{sec:gantt}

En la Figura~\ref{fig:gantt} se muestra el diagrama de Gantt con las tareas a realizar durante el desarrollo del TFM. En la tabla \ref{tab:gantt} se detallan las tareas, su duración y su fecha de inicio y fin.


\begin{figure}[H]
    \centering
    \resizebox{\linewidth}{!}{%
    \begin{ganttchart}[
    x unit=1cm,
    y unit title=0.7cm,
    y unit chart=0.5cm,
    hgrid,
    vgrid,
    bar/.append style={fill=white},
    title height=0.6,
    bar height=0.4,
    group right shift=0,
    group height=.2,
    group peaks height=.15,
    group peaks width=.5,
    group peaks tip position=0.5,
    bar label font=\small,
    group label font=\small,
    link mid=0.5
    ]
    {1}{13}
    
    \gantttitle{Trabajo de Fin de Máster}{13} \\
    \gantttitle{Sep}{1}
    \gantttitle{Oct}{1}
    \gantttitle{Nov}{1}
    \gantttitle{Dic}{1}
    \gantttitle{Ene}{1}
    \gantttitle{Feb}{1}
    \gantttitle{Mar}{1}
    \gantttitle{Abr}{1}
    \gantttitle{May}{1}
    \gantttitle{Jun}{1}
    \gantttitle{Jul}{1}
    \gantttitle{Ago}{1}
    \gantttitle{Sep}{1} \\
    
    \ganttgroup{Estudio Inicial}{1}{3} \\
    \ganttbar[bar/.append style={fill=teal!50}]{Recopilación de Datos}{1}{2} \\
    \ganttbar[bar/.append style={fill=teal!20}]{Limpieza de Datos}{3}{3} \\

    \ganttgroup{Desarrollo Investigación}{4}{12} \\
    \ganttbar[bar/.append style={fill=orange!50}]{Imputación de Datos}{4}{5} \\
    \ganttbar[bar/.append style={fill=orange!20}]{Planteamiento de Objetivos}{3}{4} \\
    \ganttbar[bar/.append style={fill=orange!20}]{Modelado de Datos}{6}{7} \\
    \ganttbar[bar/.append style={fill=purple!50}]{Interpretación y Resultados}{9}{10} \\
    \ganttbar[bar/.append style={fill=purple!20}]{Contraste de Resultados}{11}{11} \\
    \ganttbar[bar/.append style={fill=olive!50}]{Conclusiones}{12}{12} \\
    \ganttbar[bar/.append style={fill=teal!70}]{Redacción del Documento}{9}{12} \\

    \ganttgroup{Desarrollo de Código}{3}{12} \\
    \ganttbar[bar/.append style={fill=blue!40}]{Algoritmos de Preprocesamiento}{3}{6} \\
    \ganttbar[bar/.append style={fill=blue!30}]{Implementación de Modelos}{6}{7} \\
    \ganttbar[bar/.append style={fill=blue!20}]{Validación y Ajustes}{7}{9} \\
    \ganttbar[bar/.append style={fill=blue!10}]{Evaluación}{10}{12} \\
    

    \ganttlink{elem1}{elem2}
    \ganttlink{elem2}{elem4}
    \ganttlink{elem4}{elem6}
    \ganttlink{elem5}{elem6}
    \ganttlink{elem6}{elem7}
    \ganttlink{elem7}{elem8}
    \ganttlink{elem8}{elem9}
    
    \ganttlink{elem1}{elem12}
    \ganttlink{elem4}{elem13}
    \ganttlink{elem14}{elem15}

    \end{ganttchart}%
    }
    \caption{Diagrama de Gantt para el Presente Trabajo de Fin de Máster con Flechas de Dependencia y Grupos}
    \label{fig:gantt}
    \end{figure}


\begin{table}[ht]
    \label{tab:gantt}
    \centering
    \renewcommand{\arraystretch}{1.2} % Espaciado vertical en las celdas
    \begin{tabular}{p{4cm} p{2.5cm} p{2.5cm} p{2.5cm}}
    \toprule
    \textbf{Tarea} & \textbf{Duración (meses)} & \textbf{Inicio} & \textbf{Fin} \\
    \midrule
    Recopilación de Datos          & 2 & Sep & Oct \\
    Limpieza de Datos              & 1 & Nov & Nov \\
    Imputación de Datos            & 2 & Dic & Ene \\
    Planteamiento de Objetivos     & 2 & Nov & Dic \\
    Modelado de Datos              & 2 & Feb & Mar \\
    Interpretación y Resultados    & 2 & May & Jun \\
    Contraste de Resultados        & 1 & Jul & Jul \\
    Conclusiones                   & 1 & Ago & Ago \\
    Redacción del Documento        & 4 & May & Ago \\
    Algoritmos de Preprocesamiento & 4 & Nov & Feb \\
    Implementación de Modelos      & 2 & Feb & Mar \\
    Validación y Ajustes           & 3 & Mar & May \\
    Evaluación                     & 3 & Jun & Ago \\
    \bottomrule
    \end{tabular}
    \caption{Cronograma de Tareas}
    \end{table}
\newpage

%%%%%%%%%%%%%%%%%%%%%%%%%%%%%%%%%%%%%%%%%%%%%%%%%%

%%%%%%%%%%%%%%%% - BIBLIOGRAFÍA - %%%%%%%%%%%%%%%%
% Se genera la bibliografía mediante el comando \printbibliography (en ella aparecen únicamente las referencias citadas a lo largo del documento):

\appto{\bibsetup}{\sloppy}
\printbibliography[heading=bibintoc, title=BIBLIOGRAFÍA] % el argumento "title" puede modificarse indicando el título que convenga (bibliografía, referencias, etc.).
\newpage
%%%%%%%%%%%%%%%%%%%%%%%%%%%%%%%%%%%%%%%%%%%%%%%%%%


%%%%%%%%%%%%%%%%%%% - ANEXOS - %%%%%%%%%%%%%%%%%%%
\section*{ANEXOS} \label{sec:anexos} % Se añade un asterisco a \section para que el título no esté numerado.
\addcontentsline{toc}{section}{ANEXOS} % Al utilizar \section* se ha de añadir manualmente el apartado al índice (Table Of Contents, TOC).
\markright{ANEXOS} % Al utilizar \section* se ha de añadir manualmente el título del apartado al encabezado.

\renewcommand{\thesubsection}{\Alph{subsection}} % Se numeran los anexos con letras del alphabeto en lugar de números.
\setcounter{subsection}{0} % Reset the subsection counter to start from A

% Se indica que las tablas, figuras y códigos se numeran con el código del anexo (A, B, C, ...) seguido del número de tabla, figura o código dentro del anexo (tabla A.2, figura C.1, etc.)
\renewcommand{\thetable}{\Alph{subsection}.\arabic{table}}
\renewcommand{\thefigure}{\Alph{subsection}.\arabic{figure}}
\renewcommand{\thecode}{\Alph{subsection}.\arabic{code}}

% ---------------- Primer anexo ---------------- %

\subsection{PRIMER ANEXO}\label{sec:anexo1}

\subsubsection{Visualización de los datos}\label{sec:codigo-visualizacion}
\lstset{style=mystyle2} 
\begin{lstlisting}[style=mystyle2,caption={Código Visualización de los Datos}, label={lst:codigo-visualizacion}]
  ---
  title: "Interactive Graphs"
  author: "Gonzalo Aris 16021"
  date: "2023-02-11"
  output:
    html_document:
      code_folding: hide
      highlight: textmate
      toc: yes
      toc_float:
        collapsed: no
      theme: cosmo
  runtime: shiny
  resource_files:
  - "data/raw-data/ACR-11231843.xlsx"
  - data/gather_FC_all_patients.xlsx
  ---
  
  ```{r setup, include=FALSE}
  knitr::opts_chunk$set(echo = TRUE)
  ```
  
  ```{r, warning=FALSE, echo=FALSE, message=FALSE}
  ### Libraries
  library(ggplot2)
  library(readxl)
  library(dplyr)
  library(writexl)
  library(data.table)
  library(readxl)
  library(shiny)
  library(readr)
  library(astsa)
  library(xlsx)
  library(ggpubr) # ggarrange
  ```
  
  ## Graphic Information
  
  ### Heart Rate and O2Sat
  
  ```{r, echo=FALSE, message=FALSE}
  #setwd("C:/Users/User/PROYECTOS/TFM-Gonzalo-Aris/deploying")
  gather_FC_all_patients <-
    read_excel("data/gather_FC_all_patients.xlsx")
  gather_SatO2_all_patients <-
    read_excel("data/gather_SatO2_all_patients.xlsx")
  file_patient_name <- read.csv("data/file_patient_name.csv")
  file_patient_name <- file_patient_name$x 
  ```
  
  ```{r}
  selectInput('patient',
              label = 'Patient: ',
              choices = file_patient_name,
              selected = "GLR_11225596")
  
  renderText(paste0("Patient: ", as.character(input$patient)))
  
  renderPlot({
    par(mar = c(4, 4, .1, .5))
    ggplot(gather_FC_all_patients[gather_FC_all_patients$time_series == as.character(input$patient), ]) +
      aes(x =  time, y = value, color = time_series) +
      geom_line(color = "black") +
      geom_point(color = "black") +
      theme(axis.text.x = element_text(angle = 60, hjust = 1)) + labs(
        title  = "Frecuencia Cardiaca",
        x = "Time",
        y = "BPM",
        subtitle = paste0(as.character(input$patient))
      )
  })
  
  renderPlot({
    #par(mar = c(4, 4, .1, .5))
    ggplot(gather_SatO2_all_patients[gather_SatO2_all_patients$time_series == as.character(input$patient), ]) +
      aes(x =  time, y = value, color = time_series) +
      geom_line(color = "black") +
      geom_point(color = "black") +
      
      theme(axis.text.x = element_text(angle = 60, hjust = 1)) + labs(
        title  = "Sat O2",
        x = "Time",
        y = "% O2",
        subtitle = paste0(as.character(input$patient))
      )
    
  })
  ```
  
  ## Imputed Data
  
  > Data imputation based on: <https://rpubs.com/garisj98/TimeSeriesImputation>
  
  ```{r, message=FALSE, warning=FALSE}
  file_patient_name_NO_DETERIORO <-
    data.frame(read_xlsx("data/file_patient_name_NO_DETERIORO.xlsx"))
  file_patient_name_NO_DETERIORO <- file_patient_name_NO_DETERIORO$x
  file_patient_name_DETERIORO <-
    data.frame(read_xlsx("data/file_patient_name_DETERIORO.xlsx"))
  file_patient_name_DETERIORO <- file_patient_name_DETERIORO$x
  
  
  #Imputed Data
  graph_data_FC = data.frame(read_xlsx("data/graph_data_FC.xlsx", col_names = TRUE))
  graph_data_SatO2 = data.frame(read_xlsx("data/graph_data_SatO2.xlsx", col_names = TRUE))
  ```
  
  ### Heart Rate
  
  ```{r}
  renderPlot({
    #par(mar = c(4, 4, .1, .5))
    ggplot(graph_data_FC, aes(x = time, y = graph_data_FC[, as.character(input$patient)])) +
      
      geom_line(color = "black") + xlab("") +
      
      geom_point(color = ifelse(graph_data_FC[, paste0(as.character(input$patient), ".1")] == TRUE, '#69b3a2', 'black')) +
      
      theme(axis.text.x = element_text(angle = 60, hjust = 1)) +
      
      labs(title  = "Heart Rate", subtitle = paste0(as.character(input$patient)))
    
  })
  ```
  
  ### SatO2
  
  ```{r}
  renderPlot({
    #par(mar = c(4, 4, .1, .5))
    ggplot(graph_data_SatO2, aes(x = time, y = graph_data_SatO2[, as.character(input$patient)])) +
      
      geom_line(color = "black") + xlab("") +
      
      geom_point(color = ifelse(graph_data_SatO2[, paste0(as.character(input$patient), ".1")] == TRUE, '#69b3a2', 'black')) +
      
      theme(axis.text.x = element_text(angle = 60, hjust = 1)) +
      
      labs(title  = "Sat O2", subtitle = paste0(as.character(input$patient)))
    
  })
  ```
  
  
  ## Time Series interface
  
  ```{r, message=FALSE, warning=FALSE}
  file_patient_name_NO_DETERIORO <-
    data.frame(read_xlsx("data/file_patient_name_NO_DETERIORO.xlsx"))
  file_patient_name_NO_DETERIORO <- file_patient_name_NO_DETERIORO$x
  file_patient_name_DETERIORO <-
    data.frame(read_xlsx("data/file_patient_name_DETERIORO.xlsx"))
  file_patient_name_DETERIORO <- file_patient_name_DETERIORO$x
  
  
  #Imputed Data
  FC_all_valid_patients_input = data.frame(
    read_xlsx(
      "data/FC_all_valid_patients_input.xlsx",
      sheet = "FC_all_valid_patients_input",
      col_names = TRUE
    )
  )
  
  FC_all_valid_patients_Binary_Mask_NA = data.frame(
    read_xlsx(
      "data/FC_all_valid_patients_input.xlsx",
      sheet = "Binary_Mask_NA",
      col_names = TRUE
    )
  )
  
  SatO2_all_valid_patients_input = data.frame(
    read_xlsx(
      "data/SatO2_all_valid_patients_input.xlsx",
      sheet = "SatO2_all_valid_patients_input",
      col_names = TRUE
    )
  )
  
  SatO2_all_patients_Binary_Mask_NA = data.frame(
    read_xlsx(
      "data/SatO2_all_valid_patients_input.xlsx",
      sheet = "Binary_Mask_NA",
      col_names = TRUE
    )
  )
  
  time = c(1:dim(FC_all_valid_patients_input)[1])
  ```
  
  ```{r}
  
  
  radioButtons(
    "select",
    label = "Select Deterioro or Not Deterioro",
    choices = list("DETERIORO" = 1, "NO DETERIORO" = 2),
    selected = 1
  )
  
  renderUI(if (input$select == 1) {
    selectInput(
      'patient_0',
      label = 'Patient:',
      choices = file_patient_name_DETERIORO,
      selected = "APA_11204819"
    )
  } else {
    selectInput(
      'patient_0',
      label = 'Patient:',
      choices = file_patient_name_NO_DETERIORO,
      selected = "ADAO_11159808"
    )
  })
  
  ```
  
  ### ACC Heart Rate and O2Sat
  
  ```{r}
  sliderInput(
    "slider_lag_1",
    label = "Lag Max ACC",
    min = 2,
    max = 200,
    value = 50
  )
  ```
  
  ```{r}
  renderPlot({
    acf(FC_all_valid_patients_input[, as.character(input$patient_0)],
        lag.max = input$slider_lag_1,
        main = "Auto- and Cross-Correlation Hear Rate")
  })
  renderPlot({
    acf(
      SatO2_all_valid_patients_input[, as.character(input$patient_0)],
      lag.max = input$slider_lag_1,
      main = "Auto- and Cross-Correlation SatO2"
    )
  })
  ```
  
  ### CCF between Heart Rate and O2Sat
  
  ```{r}
  sliderInput(
    "slider_lag_2",
    label = "Lag Max CCF",
    min = 2,
    max = 200,
    value = 50
  )
  ```
  
  ```{r}
  renderPlot({
    ccf(
      FC_all_valid_patients_input[, as.character(input$patient_0)],
      SatO2_all_valid_patients_input[, as.character(input$patient_0)],
      lag.max = input$slider_lag_2,
      ylab = 'CCF',
      main = "Cross Correlation"
    )
  })
  ```
  
\end{lstlisting}

\lstset{style=mystyle} 

\subsubsection{Inputación con Distancia de Gower}\label{sec:codigo-input-gower}

\paragraph{Función Imputación por el Método de Gower: Input\_Gower}\label{sec:codigo-input-gower-fun}


\begin{lstlisting}[style=mystyle,caption={Input\_Gower.R}, label={lst:input-gower-fun}]
  Input_Gower <- function(dat, new_with_null, n_primeros) {
  #### SELECCIÓN DE DATOS PARA IMPUTACIÓN ###
  # Juntamos los dos data.frames. Aquel que tiene todos los pacientes sin valores faltantes: `dat` y el data.frame que contiene al paciente con valores faltantes: `new_with_null`. Se añadirá a la última fila el paciente al que se le quieren imputar datos.
  datos = rbind(dat, new_with_null)
  
  # Columnas del paciente introducido donde existen valores faltantes
  na_por_columna <- colSums(is.na(datos))
  
  # Si no hay necesidad de imputar datos se devuelve directamente al paciente y se termina el proceso.
  if (sum(na_por_columna) == 0) {
    return(new_with_null)
  }
  
  columnas_con_na <-
    names(which(na_por_columna > 0)) # Nombres de las columnas del paciente introducido con valores faltantes
  selected_columns <-
    setdiff(names(datos), columnas_con_na) # Nombres de las columnas utilizadas para calcular la dist.Gower
  datos_filtrados <-
    datos[, selected_columns] # Datos que se van a utilizar para calcular la distancia, se excluyen las columnas que tengan valores faltantes
  
  #### CÁLCULO DIST.GOWER ###
  ## Se calcula la Distancia ##
  ### En data.y se encuentra el último paciente que es al que se le van a imputar los datos
  ### En data.x todos los demás
  distance = gower.dist(data.x = datos_filtrados[-nrow(datos_filtrados), ], data.y =  datos_filtrados[nrow(datos_filtrados), ])
  distance_array = array(distance)  # La salida es un data.frame, se convierte en array
  
  ### SELECCIÓN  DE PACIENTES ###
  distance_array_ordenado <-
    sort(distance_array) # Se ordenan las distancias de menor a mayor
  n_primeros_pat <-
    head(distance_array_ordenado, n = n_primeros) # Se aíslan las primeras n_primeros distancias mas pequeñas
  indices <-
    which(distance_array %in% n_primeros_pat) # Índices de los n_primeros pacientes más cercanos
  
  datos[c(indices), ] # Estos son los n_primeros datos más cercanos que se van a usar para imputar los demás
  
  ### IMPUTAR DATOS ###
  names_df1 <-
    names(datos) # Todas las variables, entre ellas las que se necesitan imputar
  names_df2 <-
    names(datos_filtrados) # Todas las variables que no se necesitan imputar
  columnas_no_compartidas <-
    setdiff(names_df1, names_df2) # Se estudia las variables que no se comparten pues se imputaran datos en ellas
  
  ### Se estudian por separado las variables Cuantitativas de las Cualitativas
  #### Cuantitativas:
  columnas_no_compartidas_numericas <-
    setdiff(names(datos[, sapply(datos, is.numeric)]),
            names(datos_filtrados[sapply(datos_filtrados, is.numeric), ]))
  
  #### Cualitativas:
  columnas_no_compartidas_character <-
    setdiff (columnas_no_compartidas,
             columnas_no_compartidas_numericas)
  
  #### Inputación Cuantitativas con la media de los n_primeros más cercanos
  for (i in columnas_no_compartidas_numericas) {
    print(i)
    new_with_null[, i] = mean(datos[c(indices), i])
    print(mean(datos[c(indices), i]))
  }
  
  #### Inputación Cualitativas con la mediana de los n_primeros más cercanos
  for (i in columnas_no_compartidas_character) {
    print(i)
    print(names(table(datos[c(indices), i]))[which.max(table(datos[c(indices), i]))])
    new_with_null[, i] <-
      names(table(datos[c(indices), i]))[which.max(table(datos[c(indices), i]))]
  }
  
  
  return(new_with_null) # Se devuelve al paciente con los datos imputados, es una "variable fila"}
\end{lstlisting}


\newpage

\paragraph{Función de preprocesamiento: data\_imputation\_Gower}\label{sec:codigo-input-gower-fun-preprocesamiento}

\begin{lstlisting}[style=mystyle,caption={data\_imputation\_Gower.R}, label={lst:input-gower-fun-preprocesamiento}]
  data_imputation_Gower <- function(df_merge_imputation, n_primeros) {
  # Pacientes con valores faltantes
  df_with_na <-
    df_merge_imputation[!complete.cases(df_merge_imputation),]
  # Pacientes que se van a utilizar para imputar los valores faltantes
  df_without_na <-
    df_merge_imputation[complete.cases(df_merge_imputation),]
  
  for (i in c(1:dim(df_with_na)[1])) {
    df_merge_imputation[row.names(df_with_na[i, ]), ] = Input_Gower(df_without_na, df_with_na[i, ], n_primeros)
  }
  
  return(df_merge_imputation)
}
\end{lstlisting}

\vspace{-5pt}

\newpage
% ---------------- Segundo anexo --------------- %
\subsection{SEGUNDO ANEXO}~\label{sec:anexo2}

\subsubsection{Cálculo de Medias Horarias}~\label{sec:calculo-medias-horarias}
Se muestra en este Anexo el código que permite calcular la media por hora y analice la cantidad de valores faltantes y la cantidad de valores registrados en esta hora. Esta función permitirá observar varias cosas importantes para cada paciente:

\begin{enumerate}
    \item \texttt{Hora}: Tiempo referente a la recopilación de los datos.
    
    \item \texttt{N}: Cantidad de valores temporales.
    
    \item \texttt{Faltantes\_FC}: Valores de FC faltantes.
    
    \item \texttt{Faltantes\_SatO2}: Valores de SatO2 faltantes.
    
    \item \texttt{promedio\_FC\_con\_NA}: Media calculada usando los valores disponibles de FC.
    
    \item \texttt{promedio\_SatO2\_con\_NA}: Media calculada usando los valores disponibles de SatO2.
\end{enumerate}

Cada archivo se llamará \texttt{NHC-ID.xlsx} y se almacenará en \texttt{../data/info-patients/info}.Se crearán tantos archivos \texttt{.xlsx} como pacientes haya, y cada archivo tendrá el siguiente formato:

\begin{lstlisting}[style=mystyle,caption={Cálculo de Medias y Cantidad de Valores Faltantes}, label={lst:medias-y-faltantes}]
  for (name_variable in file_patient_name) {
  data = get(name_variable)
  # Count of values
  valores_unicos <- unique(data$hour)
  count_values <- data %>%
    group_by(hour) %>%
    count() %>%
    mutate(valor_orden = factor(hour, levels = valores_unicos)) %>%
    arrange(valor_orden)
  count_values <- data.frame(count_values[, c("hour", "n")])
  
  # Missing values in FC
  Missing_FC <- data %>%
    group_by(hour) %>%
    dplyr::summarize(Missing_FC = sum(is.na(FC))) %>%
    mutate(valor_orden = factor(hour, levels = valores_unicos)) %>%
    arrange(valor_orden)
  Missing_FC <- data.frame(Missing_FC[, c("Missing_FC")])
  
  # Missing values in SatO2
  Missing_SatO2 <- data %>%
    group_by(hour) %>%
    dplyr::summarize(Missing_SatO2 = sum(is.na(SatO2))) %>%
    mutate(valor_orden = factor(hour, levels = valores_unicos)) %>%
    arrange(valor_orden)
  Missing_SatO2 <- data.frame(Missing_SatO2[, c("Missing_SatO2")])
  
  
  # Mean FC
  data <- data.table(data)
  Mean_FC <- data.frame(data[, list(avg_SatFC = mean(FC)), by = hour])
  Mean_FC <- data.frame(Mean_FC[, c("avg_SatFC")])
  
  # Mean SatO2
  data <- data.table(data)
  Mean_SatO2 <-
    data.frame(data[, list(avg_SatO2 = mean(SatO2)), by = hour])
  Mean_SatO2 <- data.frame(Mean_SatO2[, c("avg_SatO2")])
  
  # Mean cuantiles.FC
  data <- data.table(data)
  Mean_cuantiles.FC <-
    data.frame(data[, list(avg_cuantiles.FC = mean(cuantiles.FC)), by = hour])
  Mean_cuantiles.FC <-
    data.frame(Mean_cuantiles.FC[, c("avg_cuantiles.FC")])
  
  # Mean scaled.FC
  data <- data.table(data)
  Mean_scaled.FC <-
    data.frame(data[, list(avg_scaled.FC = mean(scaled.FC)), by = hour])
  Mean_scaled.FC <- data.frame(Mean_scaled.FC[, c("avg_scaled.FC")])
  
  # Mean with not NA values Sat02
  Mean_SatO2_with_NA <-  data %>%
    group_by(hour) %>%
    summarise(avg_SatO2_with_NA = mean(SatO2, na.rm = T)) %>%
    mutate(valor_orden = factor(hour, levels = valores_unicos)) %>%
    arrange(valor_orden)
  Mean_SatO2_with_NA <-
    data.frame(Mean_SatO2_with_NA[, c("avg_SatO2_with_NA")])
  
  # Mean with not NA values FC
  Mean_FC_with_NA <-  data %>%
    group_by(hour) %>%
    summarise(avg_FC_with_NA = mean(FC, na.rm = T)) %>%
    mutate(valor_orden = factor(hour, levels = valores_unicos)) %>%
    arrange(valor_orden)
  Mean_FC_with_NA <-
    data.frame(Mean_FC_with_NA[, c("avg_FC_with_NA")])
  
  # Mean with not NA values cuantiles FC
  Mean_cuantiles.FC_with_NA <-  data %>%
    group_by(hour) %>%
    summarise(avg_cuantiles.FC_with_NA = mean(cuantiles.FC, na.rm = T)) %>%
    mutate(valor_orden = factor(hour, levels = valores_unicos)) %>%
    arrange(valor_orden)
  Mean_cuantiles.FC_with_NA <-
    data.frame(Mean_cuantiles.FC_with_NA[, c("avg_cuantiles.FC_with_NA")])
  
  # Mean with not NA values scaled FC
  Mean_scaled.FC_with_NA <-  data %>%
    group_by(hour) %>%
    summarise(avg_scaled.FC_with_NA = mean(scaled.FC, na.rm = T)) %>%
    mutate(valor_orden = factor(hour, levels = valores_unicos)) %>%
    arrange(valor_orden)
  Mean_scaled.FC_with_NA <-
    data.frame(Mean_scaled.FC_with_NA[, c("avg_scaled.FC_with_NA")])
  
  # Merging all the data frames
  merged_df = data.frame(
    cbind(
      count_values,
      Missing_FC,
      Missing_SatO2,
      Mean_FC,
      Mean_SatO2,
      Mean_cuantiles.FC,
      Mean_scaled.FC,
      Mean_FC_with_NA,
      Mean_SatO2_with_NA,
      Mean_cuantiles.FC_with_NA,
      Mean_scaled.FC_with_NA
    )
  )
  
  colnames(merged_df) <-
    c(
      "hour",
      "n",
      "Missing_FC",
      "Missing_SatO2",
      "Mean_FC",
      "Mean_SatO2",
      "Mean_Q",
      "Mean_SC",
      "Mean_FC_NaN",
      "Mean_SatO2_NaN",
      "Mean_Q_NaN",
      "Mean_SC_NaN"
    )
  
  assign(paste0(name_variable, "_info"), merged_df)
  paste0("../data/info-patients/", name_variable, "_info", ".xlsx")
  # For written the tables in excel files
  write_xlsx(
    merged_df,
    paste0(
      "../data/info-patients/info/",
      name_variable,
      "_info",
      ".xlsx"
    )
  )
}
\end{lstlisting}


\subsubsection{Creación de Dataframes Individuales de \textit{Frecuencia Cardíaca} y de \textit{Saturación de Oxígeno} }\label{sec:anexo2-dataframes}

\begin{lstlisting}[style=mystyle,caption={Creación de Dataframes Individuales de \textit{Frecuencia Cardíaca} y de \textit{Saturación de Oxígeno}}, label={lst:dataframes-idividuales}]
    # Lets put together all the patients in the same data frame
M <- 1441
N <- length(file_patient_name)
FC_all_patients  <- as.data.frame(x = matrix(data = NA,
                                             nrow = M,
                                             ncol = N))
SatO2_all_patients  <- as.data.frame(x = matrix(data = NA,
                                                nrow = M,
                                                ncol = N))
colnames(FC_all_patients) = file_patient_name
colnames(SatO2_all_patients) = file_patient_name


## Imputing the data inside the created data frames
for (name_file in file_patient_name) {
  data = get(name_file)
  FC_all_patients[, paste0(name_file)] <- data$FC
  SatO2_all_patients[, paste0(name_file)] <- data$SatO2
}

#Adding an extra column for the time series reference
FC_all_patients$time <- c(1:M)
SatO2_all_patients$time <- c(1:M)


write_xlsx(FC_all_patients, "../data/FC_&_SatO2/FC_all_patients.xlsx")
write_xlsx(SatO2_all_patients,
           "../data/FC_&_SatO2/SatO2_all_patients.xlsx")
\end{lstlisting}

\newpage

%%%%%%%%%%%%%%%%%%%%%%%%%%%%%%%%%%%%%%%%%%%%%%%%%%

%%%%%%%%%%%%%% - FIN DEL DOCUMENTO - %%%%%%%%%%%%%

\end{document}
