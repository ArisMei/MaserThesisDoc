\subsection{Metodología del Estudio 1}\label{sec:metodologia-estudio-1}

En este primer estudio se realizará un análisis de la varianza de las medias de los datos de monitorización de los pacientes pediátricos; los que han necesitado OAF y a los que no la han necesitado. Este análisis de la varianza de los $68$ pacientes pertenecientes del conjunto de datos \texttt{Valid\_patients\_P1} se hará para cada hora de las $24$ totales de monitorización de los pacientes.

Para dicho análisis se hará uso de la prueba $t$ de Student. Esta es una técnica estadística utilizada para comparar las medias de dos grupos (como se presenta en esta situación) y determinar si las diferencias observadas entre ellas son estadísticamente significativas. Se basa en calcular un estadístico $t$ que mide la diferencia entre las medias en términos de la variabilidad dentro de los grupos.

Supongamos que tenemos dos poblaciones, $X$ e $Y$, de las que se han obtenido muestras aleatorias simples con tamaños $n$ y $m$, respectivamente. La hipótesis nula y alternativa se plantean de la siguiente manera:

Hipótesis Nula ($H_0$):
\[
\mu_X = \mu_Y
\]

Hipótesis Alternativa ($H_1$):
\[
\mu_X \neq \mu_Y \quad \text{o} \quad \mu_X < \mu_Y \quad \text{o} \quad \mu_X > \mu_Y
\]

El estadístico de la prueba se calcula de la siguiente manera:
\[
t = \frac{\bar{X} - \bar{Y}}{s_{pooled} \cdot \sqrt{\frac{1}{n} + \frac{1}{m}}},
\]
donde:
\begin{align*}
\bar{X} & \text{ es la media del grupo } X \\
\bar{Y} & \text{ es la media del grupo } Y \\
s_{pooled} & \text{ es la desviación estándar agrupada} \\
n & \text{ es el tamaño del grupo } X \\
m & \text{ es el tamaño del grupo } Y
\end{align*}

{\color{blue} El uso de $s_{pooled}$ y la distribución t de Student con $n+m-2$ asume que las varianzas sean iguales.}

El estadístico $t$ calculado se compara con un valor crítico de la distribución t de Student con $n+m-2$ grados de libertad para determinar si las diferencias entre las medias son estadísticamente significativas. Si el valor $p$ asociado con el estadístico $t$ es menor que el umbral predefinido (nivel de significancia $\alpha$), se rechaza la hipótesis nula y se concluye que hay una diferencia significativa entre las medias.

Este estudio se va a realizar para la \textit{Frecuencia Cardiaca}, para la \textit{Frecuencia Cardiaca Transformada por Cuantiles}, para la \textit{Frecuencia Cardiaca Escalada} y para la \textit{Saturación de Oxígeno Escalada}. Se quiere pues observar si realmente existe una diferencia significativa entre las medias de los datos de monitorización de los pacientes pediátricos que han necesitado OAF y los que no la han necesitado.

Para ello partiendo de los $1440$ valores de monitorización se calcularán las medias horarias gracias al código mostrado en el Anexo~\ref{sec:anexo2}.



