\section{CAPÍTULO 8: RESPONSABILIDAD SOCIAL Y PROFESIONAL}\label{cap:responsabilidadSocialYProfesional}

\subsection{Evaluación de impacto social, ambiental, legal y ético}

\subsubsection{Impacto Social}

El impacto social de este proyecto es muy positivo, se permitirá la detección de pacientes de riesgo y que se tomarán decisiones clínicas tempranas a la hora de suministrar OAF. Se mejorará pues la atención médica para los pacientes pediátricos con cuadro bronquiolitis afectando positivamente así a la sociedad. 

\subsubsection{Impacto Ambiental}

El impacto ambiental de este proyecto es nulo, ya que no se ha utilizado ningún material o técnica que pueda afectar al medio ambiente más allay de las herramientas y programas ofimáticos convenientes para el desarrollo del proyecto.

\subsubsection{Impacto Legal}

El impacto legal de este proyecto va allí hasta dónde se asegura que los datos de los pacientes han sido anonimizados y sus tutores legales han dado su consentimiento para que se utilicen los datos de sus hijos para la realización de este proyecto.

Una vez realizado el presente trabajo de investigación existe compromiso por parte de los que han sido involucrados en proporcionar transparencia en la metodología y el funcionamiento del modelo. 

\subsubsection{Impacto Ético}

En relación al impacto ético de este proyecto se asegura que los pacientes y sus tutores legales estén debidamente informados y den su consentimiento para participar en este trabajo. Así como se asegura la distribución equitativa de los beneficios y cargas de este proyecto.
\newpage

\subsection{Contribución a los Objetivos de Desarrollo Sostenible y Agenda 2030}

La contribución a los Objetivos de Desarrollo Sostenible (ODS) y la Agenda 2030 es un aspecto fundamental que debe ser considerado en cualquier proyecto, incluido uno que aborde la predicción de suministro de OAF en pacientes pediátricos. A continuación, se describen algunas formas en que este proyecto podría contribuir a los ODS y la Agenda 2030:

\begin{itemize}
    \item \textsc{ODS 3: Salud y bienestar.} Este proyecto contribuye a este objetivo ya que se pretende mejorar la atención médica de los pacientes pediátricos con cuadro bronquiolitis mejorando su salud y bienestar
    \item \textsc{ODS 9: Industria, innovación e infraestructura.} Este proyecto contribuye a este objetivo ya que parte de la metodología utilizada para la realización de este proyecto se basa en la utilización de técnicas de aprendizaje automático, que es una de las tecnologías más innovadoras de la actualidad.
    \item \textsc{ODS 17: Alianzas para lograr los objetivos.} Este proyecto contribuye a este objetivo ya que se ha realizado en colaboración con el Hospital Universitario de Gregorio Marañón y la Universidad Politécnica de Madrid.
    \item \textsc{ODS 16: Paz, justicia e instituciones sólidas.} Este proyecto contribuye a este objetivo ya que a la hora de la recopilación y el uso de datos de pacientes se siguen un estándares éticos y y se pretende asegurar el respeto de los derechos de los pacientes.
\end{itemize}

