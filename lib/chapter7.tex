\section{PLANIFICACIÓN Y PRESUPUESTO}\label{cap:results}

\subsection{Planificación}\label{sec:planificacion}

\subsubsection{Estructura de la Descomposición del Trabajo de Fin de Máster}\label{sec:edt}

La estructura de la descomposición del trabajo de fin de máster se muestra en la Figura \ref{fig:tfg_structure}. Se ha dividido en tres grandes bloques: \textit{Estudio Inicial}, \textit{Planteamiento y Modelado} y \textit{Redacción y Conclusiones}. 

El primer bloque se centra en la recopilación, limpieza e imputación de los datos. El segundo bloque se centra en el planteamiento de los objetivos, el modelado de los datos, la interpretación de los resultados y el contraste de los mismos. El tercer bloque se centra en la redacción del informe, la introducción, el desarrollo de la metodología, los resultados y el análisis, las conclusiones, las contribuciones y la revisión y edición del informe.

\thispagestyle{empty}
\newgeometry{top=10mm, bottom=10mm, left=12mm, right=12mm}
\begin{landscape}

\begin{figure}
\centering
\begin{tikzpicture}[scale=0.9,
every node/.style = {draw, rounded corners=3pt, semithick, drop shadow, font=\small},
ROOT/.style = {color=gray!70, text width = 14.5cm, inner sep=2mm, text=white, font=\bfseries, text centered},
L1/.style = {fill=white, text width = 3.5cm},
L2/.style = {fill=white, text width = 3cm, grow=down, xshift=1em, anchor=west, edge from parent path={(\tikzparentnode.south) |- (\tikzchildnode.west)}},
edge from parent/.style = {draw, thick},
LD/.style = {level distance=#1ex},
level 1/.style = {sibling distance=90mm}
]

\node[ROOT] {Trabajo de Fin de Máster}
[edge from parent fork down]
    child[LD=14] {node[L1, text width=6cm] {1. Estudio Inicial}
        child[L2,LD=9]   {node[L2]   {1.1 Recopilación de Datos}}
        child[L2,LD=18]  {node[L2]   {1.2 Limpieza de Datos}}
        child[L2,LD=28]   {node[L2]   {1.3 Imputación de Datos}
            child[L2,LD=11] {node[L2]   {1.3.1 Métodos de Imputación}}
            child[L2,LD=21] {node[L2]   {1.3.2 Evaluación de Imputaciones}}
            child[L2,LD=32] {node[L2]   {1.3.3 Selección de Imputaciones Finales}}
            }
        }
    child[LD=14]{node[L1, LD=40] {2. Planteamiento y Modelado}
        child[L2,LD=9]  {node[L2, text width=4.2cm]   {2.1 Planteamiento de Objetivos}
            child[L2,LD=9] {node[L2, text width = 3.5cm]   {2.1.1 Definición de Problema}}
            child[L2,LD=18] {node[L2, text width = 3.5cm]   {2.1.2 Determinación de Metas}}
            }
        child[L2,LD=37]  {node[L2, text width=4.3cm]   {2.2 Modelado de Datos}
            child[L2,LD=11] {node[L2, text width = 3.5cm]   {2.2.1 Selección de Modelos}}
            child[L2,LD=23] {node[L2, text width = 3.5cm]   {2.2.2 Entrenamiento de Modelos}}
            child[L2,LD=34] {node[L2, text width = 3.5cm]   {2.2.3 Evaluación de Modelos}}
            }
        child[L2,LD=81]  {node[L2, text width=5cm]   {2.3 Interpretación y Resultados}}
        child[L2,LD=89]  {node[L2, text width=5cm]   {2.4 Contraste de Resultados}}
        child[L2,LD=96]  {node[L2, text width=3.5cm]   {2.5 Limitaciones y Futuras Investigaciones}}
        }
    child [LD=14]{node[L1, LD=40] {3. Redacción y Conclusiones}
        child[L2,LD=9] {node[L2, text width=3.3cm]   {3.1 Estructura del Informe}}
        child[L2,LD=19] {node[L2, text width=3.3cm]   {3.2 Introducción}}
        child[L2,LD=29] {node[L2, text width=4.6cm]   {3.3 Desarrollo de la Metodología}}
        child[L2,LD=39] {node[L2, text width=4.6cm]   {3.4 Resultados y Análisis}}
        child[L2,LD=49] {node[L2, text width=4.6cm]   {3.5 Conclusiones}}
        child[L2,LD=59] {node[L2, text width=4.6cm]   {3.6 Contribuciones}}
        child[L2,LD=69] {node[L2]   {3.7 Revisión y Edición}}
    };
        
\end{tikzpicture}
\vspace{10pt}
\caption{Estructura para Trabajo de Fin de Máster (TFM)}
\label{fig:tfg_structure}
\end{figure}
\end{landscape}
\restoregeometry

\subsubsection{Diagrama de Gantt}\label{sec:gantt}

En la Figura~\ref{fig:gantt} se muestra el diagrama de Gantt con las tareas a realizar durante el desarrollo del TFM. En la tabla \ref{tab:gantt} se detallan las tareas, su duración y su fecha de inicio y fin.

\begin{figure}[H]
    \centering
    \resizebox{\linewidth}{!}{%
    \begin{ganttchart}[
    x unit=1cm,
    y unit title=0.7cm,
    y unit chart=0.5cm,
    hgrid,
    vgrid,
    bar/.append style={fill=white},
    title height=0.6,
    bar height=0.4,
    group right shift=0,
    group height=.2,
    group peaks height=.15,
    group peaks width=.5,
    group peaks tip position=0.5,
    bar label font=\small,
    group label font=\small,
    link mid=0.5
    ]
    {1}{13}
    
    \gantttitle{Trabajo de Fin de Máster}{13} \\
    \gantttitle{Sep}{1}
    \gantttitle{Oct}{1}
    \gantttitle{Nov}{1}
    \gantttitle{Dic}{1}
    \gantttitle{Ene}{1}
    \gantttitle{Feb}{1}
    \gantttitle{Mar}{1}
    \gantttitle{Abr}{1}
    \gantttitle{May}{1}
    \gantttitle{Jun}{1}
    \gantttitle{Jul}{1}
    \gantttitle{Ago}{1}
    \gantttitle{Sep}{1} \\
    
    \ganttgroup{\textbf{Duración Total}}{1}{13} \\
    \ganttgroup{\textbf{Estudio Inicial}}{1}{3} \\
    \ganttbar[bar/.append style={fill=teal!50}]{Recopilación de Datos}{1}{2} \\
    \ganttbar[bar/.append style={fill=teal!20}]{Limpieza de Datos}{3}{3} \\

    \ganttgroup{\textbf{Desarrollo Investigación}}{4}{12} \\
    \ganttbar[bar/.append style={fill=orange!50}]{Imputación de Datos}{4}{5} \\
    \ganttbar[bar/.append style={fill=orange!20}]{Planteamiento de Objetivos}{3}{4} \\
    \ganttbar[bar/.append style={fill=orange!20}]{Modelado de Datos}{6}{7} \\
    \ganttbar[bar/.append style={fill=purple!50}]{Interpretación y Resultados}{9}{10} \\
    \ganttbar[bar/.append style={fill=purple!20}]{Contraste de Resultados}{11}{11} \\
    \ganttbar[bar/.append style={fill=olive!50}]{Conclusiones}{12}{12} \\
    \ganttbar[bar/.append style={fill=green!10}]{Redacción del Documento}{9}{12} \\

    \ganttgroup{\textbf{Desarrollo de Código}}{3}{12} \\
    \ganttbar[bar/.append style={fill=blue!40}]{Algoritmos de Preprocesamiento}{3}{6} \\
    \ganttbar[bar/.append style={fill=blue!30}]{Implementación de Modelos}{6}{7} \\
    \ganttbar[bar/.append style={fill=blue!20}]{Validación y Ajustes}{7}{9} \\
    \ganttbar[bar/.append style={fill=blue!10}]{Evaluación}{10}{12} \\
    \ganttbar[bar/.append style={fill=teal!70}]{Código \LaTeX}{3}{12} \\
    

    \ganttlink{elem2}{elem3}
    \ganttlink{elem3}{elem5}
    \ganttlink{elem5}{elem7}
    \ganttlink{elem6}{elem7}
    \ganttlink{elem7}{elem8}
    \ganttlink{elem8}{elem9}
    \ganttlink{elem9}{elem10}
    
    \ganttlink{elem2}{elem13}
    \ganttlink{elem5}{elem14}
    \ganttlink{elem15}{elem16}

    \end{ganttchart}%
    }
    \caption{Diagrama de Gantt para el Presente Trabajo de Fin de Máster con Flechas de Dependencia y Grupos}
    \label{fig:gantt}
    \end{figure}


\begin{table}[H]
    \centering
    \renewcommand{\arraystretch}{1.2} % Espaciado vertical en las celdas
    \begin{tabular}{p{4cm} p{2.5cm} p{2.5cm} p{2.5cm}}
    \toprule
    \textbf{Tarea} & \textbf{Duración (meses)} & \textbf{Inicio} & \textbf{Fin} \\
    \midrule
    Recopilación de Datos          & 2 & Sep & Oct \\
    Limpieza de Datos              & 1 & Nov & Nov \\
    Imputación de Datos            & 2 & Dic & Ene \\
    Planteamiento de Objetivos     & 2 & Nov & Dic \\
    Modelado de Datos              & 2 & Feb & Mar \\
    Interpretación y Resultados    & 2 & May & Jun \\
    Contraste de Resultados        & 1 & Jul & Jul \\
    Conclusiones                   & 1 & Ago & Ago \\
    Redacción del Documento        & 4 & May & Ago \\
    Algoritmos de Preprocesamiento & 4 & Nov & Feb \\
    Implementación de Modelos      & 2 & Feb & Mar \\
    Validación y Ajustes           & 3 & Mar & May \\
    Evaluación                     & 3 & Jun & Ago \\
    \bottomrule
    \end{tabular}
    \caption{Cronograma de Tareas}\label{tab:gantt}
    \end{table}

Considerando que el proyecto ha durado un año y se han invertido una media de $15$ horas semanales el textbf{Total de Horas Dedicadas} son unas $783$ horas. 



\subsection{Presupuesto}\label{sec:presupuesto}

En relación al presente Trabajo de Fin de Máster, no se han consumido recursos bibliográficos de pago y las herramienta principales utilizadas para correr y programar los códigos del proyecto, \textit{RStudio} y \LaTeX, son de uso y licencia gratuita. Por ende, sólo se tendrá en cuenta la dedicación del alumno, ciertas licencias de programas necesarios para su desarrollo, la matrícula de la universidad y los costes derivados de trabajo.

En lo que respecta a los costos relacionados con los recursos humanos asignados a esta labor, se presentarán en la Tabla~\ref{tab:costes-rrhh}. Se ha considerado un coste de $15$ euros por hora para los médicos que han recopilado los datos y un coste de $10$ euros por hora para el alumno. Además, se ha considerado un coste de $20$ euros por hora para las asesorías de los tutores y de los médicos. Por último, se ha considerado un coste de $10$ euros por hora para la corrección y supervisión de los tutores del TFM.

\begin{table}[H]
    \centering
    \begin{tabular}{p{5cm}ccc}
    \toprule
    Actividad & Horas Trabajadas & €/H & Total \\
    \midrule
    Recopilación de los datos por parte de los médicos & 200 & 15 & 3000 € \\
    Trabajo Gonzalo Aris Jiménez & 625 & 10 & 6250 € \\
    Trabajo asesoría Médico Rosa Rodríguez & 10 & 20 & 200 €  \\
    Trabajo asesoría Médico Felipe Gonzalez Fernández & 20 & 20 & 400 € \\
    Trabajo asesoría tutor TFM Carolina García & 20 & 20 &  400 € \\
    Trabajo asesoría tutor TFM Andres Modesto & 20 & 20 &  400 € \\
    Corrección y supervisión tutores TFG & 10 & 20 & 200 € \\
    \midrule
    Total & & & 10850 € \\
    \bottomrule
    \end{tabular}
    \caption{Horas Trabajadas y Costos}\label{tab:costes-rrhh}
 \end{table}


 A parte de los costes asociados a los recursos humanos se consideran aquellos costes unitarios y costes de los activos utilizados en relación al tiempo que dura el presente Trabajo de Fin de Máster.

 \begin{table}[H]
    \centering
    \begin{tabular}{p{5cm}cc>{\centering\arraybackslash}p{2.5cm}>{\centering\arraybackslash}p{2.5cm}}
    \toprule
    Actividad & Tipo & Tiempo & Coste Unitario & Total \\
    \midrule
    Licencia Office & Temporal & Anual & 100 €/año & 100 € \\
    Matrícula TFM & Temporal & Anual & 24,55 €/ECT & 294,60 € \\
    Internet y luz & Temporal & Mensual & 37,5 €/mes & 450 € \\
    Portátil \textit{HP ZBook 15 G3} & No Temporal & & & 400 € \\
    Monitor Suplementario: \textit{HP Compaq LE2202x} & No Temporal & & & 110 € \\
    Libro: \textit{Time Series Analysis with Applications in R (second edition)} & No Temporal & & & 116 € \\
    Amortización Acumulada 1er año & & & -250 € \\
    \midrule
    Total & & & & 1220,6 € \\
    \bottomrule
    \end{tabular}
    \caption{Tiempo y Costos Asociados}\label{tab:costes-unitarios-activos}
\end{table}

Los costes activos se amortizarán en un periodo de 4 años, la amortización se realizará de forma acelerada es decir las cuotas serán $0.4$, $0.3$, $0.2$, $0.1$ sobre el valor inicial. Actualmente solo se ha amortizado la primera tasa ya que el proyecto ha tenido una duración anual, quedarían tres tasas todavía a pagar de los diferentes activos comprados.

Estos activos se amortizarán para hacer frente al desgaste al que están sometidos, amortizar los activos permite reflejar el costo del uso de un activo de manera más real a su vez la amortización es una buena praxis de cara a cumplir con los principios contables de este proyecto. Las amortizaciones se muestran en la Tabla~\ref{tab:costos-amortizados}.

\begin{table}[htbp]
    \centering
    \begin{tabular}{p{4cm}ccccc}
    \toprule
    Item & Coste Unitario & 1ª Tasa & 2ª Tasa & 3ª Tasa & 4ª Tasa \\
    \midrule
    Portátil \textit{HP ZBook 15 G3} & 400 € & 160 € & 120 € & 80 € & 40 € \\
    Monitor Suplementario: \textit{HP Compaq LE2202x} & 110 € & 44 € & 33 € & 22 € & 11 € \\
    Libro: \textit{Time Series Analysis with Applications in R (second edition)} & 116 € & 46 € & 35 € & 23 € & 12 € \\
    \midrule
    Total & & 250 € & 185 € & 125 € & 63 €  \\
    \bottomrule
    \end{tabular}
    \caption{Amortizaciones Costos Activos}\label{tab:costos-amortizados}
\end{table}

Una vez expuestos los diversos costos incurridos y las amortizaciones aplicadas a los activos, se llega a la conclusión de que el \textbf{Coste Total} del presente Trabajo de Fin de Máster asciende a $10850 + 1220.6 = 12070.6$ euros.

\newpage