\section{CAPÍTULO IV: METODOLOGÍA}\label{sec:methodology}

En este capítulo, se describirá la metodología utilizada para conseguir cumplir los objetivos establecidos. 

Se comenzará por discutir el diseño de la investigación y los tipos de variables que serán utilizados a la hora de realizar el estudio. Luego se describirá el enfoque aportado para el preprocesamiento de datos, incluida la limpieza de datos y la selección de características.

Finalmente, se describirán los algoritmos de aprendizaje automático utilizados para el estudio y clasificación de pacientes. 

\subsection{Diseño de Investigación}

Se ha utilizado un enfoque de investigación cuantitativa para examinar la progresión temporal de la bronquiolitis en pacientes pediátricos. De cara a alcanzar los objetivos planteados en la Sección~\ref{sec:objectives} se discutirá como la muestra limita el cumplimiento de algunos de los objetivos y las alternativas planteadas. 

Como se ha mencionado anteriormente se van a realizar $2$ estudios diferentes con distinto enfoque. A continuación se explicarán por qué se ha decidido realizar 2 estudios diferentes y se discutirán como se deben de replantear los objetivos plantados en la Sección~\ref{sec:objectives}.

Al tener 3 momentos de valoración, que son los que se muestran a continuación:

\begin{enumerate}
    \item $0\:horas <= Tiempo de Moitorización < 8\:horas$ 
    \item $8\:horas <= Tiempo de Moitorización < 16\:horas$
    \item $16\:horas <= Tiempo de Moitorización < 24\:horas$
\end{enumerate}

Se ha plantado que se realicen valoraciones del paciente monitorizado antes de terminar el intervalo de monitorización. Es decir que al final de las $8$, $16$ y $24$ horas, se valore si: 

\begin{itemize}
    \item ¿El paciente va a necesitar OAF?
    \item ¿El paciente necesitará ser ingresado en la UCIP?
\end{itemize}

Existe un factor limitante a la hora de hacer estos tres estudios predictivos.

En primer lugar es la falta de datos de pacientes que han ingresado en la UCIP. Si se parte de los pacientes válidos \texttt{Valid\_patients\_P1}, definidos en la Sección~\ref{sec:seleccion_pacientes}, que son aquellos que muestran un porcentaje de valores faltantes menor del $5\%$ en las primeras $24$ horas de monitorización, se puede ver cómo solo $4$ han sido llevados a la UCIP, por el contrario existe un mayor registro de pacientes que han necesitado soporte respiratorio mediante el uso de la OAF, $14$. Esta distribución de pacientes se muestra en la Figura~\ref{fig:bar-OAF-UCIP-valid-1} de a continuación: 

\begin{figure}[H]
    \centering
    \includegraphics[scale = 0.9]{./img/bar-OAF-UCIP-valid-1.png}
    \caption{Cantidad de pacientes que han sido trasladados a la UCIP y los que NO, que han necesitado OAF y lo que NO, del conjunto de pacientes válidos: \texttt{valid\_patient\_1}}
    \label{fig:bar-OAF-UCIP-valid-1}
\end{figure}

De cara a plantear la investigación teniendo en cuenta los diferentes intervalos antes mencionados, es necesario ver la distribución de los pacientes dentro de los mismos. Esto se puede ver a continuación en la siguiente Figura~\ref{fig:intervalos-valid-1}.


\begin{figure}[H]
    \centering
    \includegraphics[scale = 0.9]{./img/intervalos-valid-1.png}
    \caption{Distribución de los pacientes dentro de los $3$ intervalos de estudio del conjunto de pacientes: \texttt{valid\_patient\_1}}
    \label{fig:intervalos-valid-1}
\end{figure}








Una vez pasadas las $8$ primeas horas se pretende valorar si el paciente va a necesitar OAF en las próximas $8$ horas y así sucesivamente. Partiendo de esta situación el objetivo sería que dentro de las primeras $24$ h de ingreso del paciente se le valore en $3$ momentos con qué porcentaje va a necesitar OAF en las $8$ horas siguientes. De cara a plantear esta metodología para este estudio existe el factor limitante de la muestra de pacientes con la que se trabaja, dónde la mayoría de pacientes no sufren OAF y por tanto no se puede realizar un estudio de este tipo dónde se divide por intervalos a la muestra de pacientes que sufren OAF. A continuación, se muestra en relación al conjunto de pacientes \texttt{Valid\_patients\_P1} cómo se distribuyen los pacientes que han necesitado OAF en los distintos intervalos:

En este estudio este planteamiento se muestra limitado puesto que más de la mitad de pacientes que se han considerado válidos para el estudio (según el criterio valid\_patient\_2 adoptado en el apartado ***) que necesitan OAF la requieren al ingreso o en las primeras 8 horas (De los 16 pacientes válidos que presentan deterioro 10 pacientes; 5 al ingreso y 5 antes de las 8 primeras horas). Esto supondría eliminar de la ecuación estos 10 pacientes y solo tener en cuenta a los 6 restantes para valorar las 8 primeras horas y solo a 2 para valorar las 16 primeras.



\begin{itemize}
    \item El primer estudio, \textit{Estudio 1}, como se ha tratado anteriormente en la Sección~\ref{sec:metodologias_exclusion_pacientes} de selección de pacientes válidos para los distintos estudios, se va a centrar en el estudio de todos los pacientes pediátricos considerando las primeras $24$ horas desde el ingreso. En este caso se valorará de manera general la evolución de los pacientes y diferencias entre aquellos que se les ha intervenido con OAF y los que no. En este estudio simplemente se tendrán en consideración las variables \textit{Series Temporales} referenciadas en la tabla~\ref{tabla:variables_estudio} y se realizará un estudio descriptivo de las mismas.
    \item  En el segundo estudio, \textit{Estudio 2}, se valorará la evolución de los pacientes que han necesitado OAF y se comparará con los que no han necesitado OAF para así poder generar un modelo que permita predecir si un paciente va a necesitar OAF o no en las siguientes próximas horas. Es este estudio se pretenderá realizar un modelo que permita prever la necesidad de OAF en las 8 h de ingreso. ad de OAF en las horas siguientes de las 
\end{itemize}

Se utilizarán aquellas variables que se han definido anteriormente en la Sección~\ref{sec:tiposdevariables} y en concreto de aquellas recogidas en la Tabla~\ref{tabla:variables_estudio}:

\begin{itemize}
    \item Descriptivas dentro del \textit{scope} del estudio.
    \item Temporales en $3$ Intervalos dentro del \textit{scope}.
    \item Series Temporales.
\end{itemize}

Las variables utilizadas e el \textit{Estudio 2} será menos que las usadas en el \textit{Estudio 1}

ha partido de 47 variables descriptivas y de 2 variables que muestran la evolución temporal de los pacientes durante las primeras 24 h de ingreso y que han sido tratadas en el presente trabajo como series temporales. 


Explicación de los distintos Estudios:


En el caso del Estudio 2:

Idealmente 



\begin{figure}[H]
    \centering
    \includegraphics[scale = 1]{./img/bar-deterioro-valid-2.png}
    \caption{Cantidad de Pacientes Deterioro en Rangos Específicos de Tiempo considerando solo los pacientes sgún criterio valid\_patient\_2}
    \label{fig:bar-deterioro-valid-2}
\end{figure}

Se va a separar entre pacientes que han necesitado OAF al ingreso y los que no.

A continuación se muestran en rojo aquellos pacientes que a pesar de haberles suministrado OAF han necesitado ser trasladados a UCIP.

En contraposición se muestran en verde aquellos pacientes que han necesitado OAF pero no han necesitado ser trasladados a UCIP. 



Ningún paciente cómo ya se ha visto ha necesitado traslado a UCIP sin haber necesitado OAF previamente.

