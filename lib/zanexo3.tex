\subsection{TERCER ANEXO}~\label{sec:anexo3}

\subsubsection{Código para Resultados Estudio 2}

\paragraph{Obtención de clusters}

\begin{itemize}
    \item \texttt{01-HR}: \url{https://rpubs.com/garisj98/1077202}
    \item \texttt{02-HRScaled}: \url{https://rpubs.com/garisj98/1077203}
    \item \texttt{03-HRCuant}: \url{https://rpubs.com/garisj98/1077204}
    \item \texttt{04-SatO2:} \url{https://rpubs.com/garisj98/1077205}
    \item \texttt{05-SatO2Scaled}: \url{https://rpubs.com/garisj98/1077206}
    \item \texttt{06-Bivariant}: \url{https://rpubs.com/garisj98/1077207}
\end{itemize}

\paragraph{Adjusted Rand Index}

\begin{lstlisting}[style=mystyle,caption={Adjusted Rand Index}, label={lst:input-gower-fun-preprocesamiento}]
    cluster_study_SatO2 <-
  read.csv("../../../data/clusters/cluster_study_SatO2.csv")
cluster_study_FC <- read.csv("../../../data/clusters/cluster_study_FC.csv")
cluster_study_CCF <- read.csv("../../../data/clusters/cluster_study_CCF.csv")
cluster_study_FC_scaled <- read.csv("../../../data/clusters/cluster_study_FC_scaled.csv")
cluster_study_cuantiles <- read.csv("../../../data/clusters/cluster_study_cuantiles.csv")

cluster_study <-
  data.frame(
    rbind(
      cluster_study_SatO2[[2]],
      cluster_study_SatO2[[3]],
      cluster_study_SatO2[[4]],
      cluster_study_FC[[2]],
      cluster_study_FC[[3]],
      cluster_study_FC[[4]],
      cluster_study_CCF[[2]],
      cluster_study_FC_scaled[[2]],
      cluster_study_FC_scaled[[3]],
      cluster_study_FC_scaled[[4]],
      cluster_study_cuantiles[[2]],
      cluster_study_cuantiles[[3]],
      cluster_study_cuantiles[[4]]
    )
  )

names(cluster_study) <- cluster_study_SatO2[[1]]
row.names(cluster_study) <-
  c(
    "ACF_SatO2",
    "EUCL_SatO2",
    "PER_SatO2",
    "ACF_HR",
    "EUCL_HR",
    "PER_HR",
    "CCF",
    "ACF_s_HR",
    "EUCL_s_HR",
    "PER_s_HR",
    "ACF_c_HR",
    "EUCL_c_HR",
    "PER_c_HR"
  )
write.csv(cluster_study,"../../../data/clusters/cluster_study.csv")

rand_index_table = data.frame(matrix(ncol = 13 , nrow = 13))
colnames(rand_index_table) <- c("ACF_SatO2", "EUCL_SatO2", "PER_SatO2","ACF_HR", "EUCL_HR", "PER_HR", "CCF", "ACF_s_HR", "EUCL_s_HR", "PER_s_HR","ACF_c_HR", "EUCL_c_HR", "PER_c_HR")
rownames(rand_index_table) <- c("ACF_SatO2", "EUCL_SatO2", "PER_SatO2","ACF_HR", "EUCL_HR", "PER_HR", "CCF", "ACF_s_HR", "EUCL_s_HR", "PER_s_HR","ACF_c_HR", "EUCL_c_HR", "PER_c_HR")

for (i in c(1:dim(cluster_study)[1])) {
  for (j in c(1:dim(cluster_study)[1])){
  rand_index_table[i,j] <- adjustedRandIndex(as.integer(cluster_study[i,]), as.integer(cluster_study[j,]))
}}

formattable(rand_index_table, list(area(col = ACF_SatO2:PER_c_HR) ~ color_tile("transparent", "lightblue")))
\end{lstlisting}


\paragraph{Código para el modelo final}
\textbf{FINAL MODEL CODE:} \url{https://rpubs.com/garisj98/1077280}