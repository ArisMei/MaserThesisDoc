\subsection{SEGUNDO ANEXO}~\label{sec:anexo2}

\subsubsection{Cálculo de Medias Horarias}~\label{sec:calculo-medias-horarias}
Se muestra en este Anexo el código que permite calcular la media por hora y analice la cantidad de valores faltantes y la cantidad de valores registrados en esta hora. Esta función permitirá observar varias cosas importantes para cada paciente:

\begin{enumerate}
    \item \texttt{Hora}: Tiempo referente a la recopilación de los datos.
    
    \item \texttt{N}: Cantidad de valores temporales.
    
    \item \texttt{Faltantes\_FC}: Valores de FC faltantes.
    
    \item \texttt{Faltantes\_SatO2}: Valores de SatO2 faltantes.
    
    \item \texttt{promedio\_FC\_con\_NA}: Media calculada usando los valores disponibles de FC.
    
    \item \texttt{promedio\_SatO2\_con\_NA}: Media calculada usando los valores disponibles de SatO2.
\end{enumerate}

Cada archivo se llamará \texttt{NHC-ID.xlsx} y se almacenará en \texttt{../data/info-patients/info}.Se crearán tantos archivos \texttt{.xlsx} como pacientes haya, y cada archivo tendrá el siguiente formato:

\begin{lstlisting}[style=mystyle,caption={Cálculo de Medias y Cantidad de Valores Faltantes}, label={lst:medias-y-faltantes}]
  for (name_variable in file_patient_name) {
  data = get(name_variable)
  # Count of values
  valores_unicos <- unique(data$hour)
  count_values <- data %>%
    group_by(hour) %>%
    count() %>%
    mutate(valor_orden = factor(hour, levels = valores_unicos)) %>%
    arrange(valor_orden)
  count_values <- data.frame(count_values[, c("hour", "n")])
  
  # Missing values in FC
  Missing_FC <- data %>%
    group_by(hour) %>%
    dplyr::summarize(Missing_FC = sum(is.na(FC))) %>%
    mutate(valor_orden = factor(hour, levels = valores_unicos)) %>%
    arrange(valor_orden)
  Missing_FC <- data.frame(Missing_FC[, c("Missing_FC")])
  
  # Missing values in SatO2
  Missing_SatO2 <- data %>%
    group_by(hour) %>%
    dplyr::summarize(Missing_SatO2 = sum(is.na(SatO2))) %>%
    mutate(valor_orden = factor(hour, levels = valores_unicos)) %>%
    arrange(valor_orden)
  Missing_SatO2 <- data.frame(Missing_SatO2[, c("Missing_SatO2")])
  
  
  # Mean FC
  data <- data.table(data)
  Mean_FC <- data.frame(data[, list(avg_SatFC = mean(FC)), by = hour])
  Mean_FC <- data.frame(Mean_FC[, c("avg_SatFC")])
  
  # Mean SatO2
  data <- data.table(data)
  Mean_SatO2 <-
    data.frame(data[, list(avg_SatO2 = mean(SatO2)), by = hour])
  Mean_SatO2 <- data.frame(Mean_SatO2[, c("avg_SatO2")])
  
  # Mean cuantiles.FC
  data <- data.table(data)
  Mean_cuantiles.FC <-
    data.frame(data[, list(avg_cuantiles.FC = mean(cuantiles.FC)), by = hour])
  Mean_cuantiles.FC <-
    data.frame(Mean_cuantiles.FC[, c("avg_cuantiles.FC")])
  
  # Mean scaled.FC
  data <- data.table(data)
  Mean_scaled.FC <-
    data.frame(data[, list(avg_scaled.FC = mean(scaled.FC)), by = hour])
  Mean_scaled.FC <- data.frame(Mean_scaled.FC[, c("avg_scaled.FC")])
  
  # Mean with not NA values Sat02
  Mean_SatO2_with_NA <-  data %>%
    group_by(hour) %>%
    summarise(avg_SatO2_with_NA = mean(SatO2, na.rm = T)) %>%
    mutate(valor_orden = factor(hour, levels = valores_unicos)) %>%
    arrange(valor_orden)
  Mean_SatO2_with_NA <-
    data.frame(Mean_SatO2_with_NA[, c("avg_SatO2_with_NA")])
  
  # Mean with not NA values FC
  Mean_FC_with_NA <-  data %>%
    group_by(hour) %>%
    summarise(avg_FC_with_NA = mean(FC, na.rm = T)) %>%
    mutate(valor_orden = factor(hour, levels = valores_unicos)) %>%
    arrange(valor_orden)
  Mean_FC_with_NA <-
    data.frame(Mean_FC_with_NA[, c("avg_FC_with_NA")])
  
  # Mean with not NA values cuantiles FC
  Mean_cuantiles.FC_with_NA <-  data %>%
    group_by(hour) %>%
    summarise(avg_cuantiles.FC_with_NA = mean(cuantiles.FC, na.rm = T)) %>%
    mutate(valor_orden = factor(hour, levels = valores_unicos)) %>%
    arrange(valor_orden)
  Mean_cuantiles.FC_with_NA <-
    data.frame(Mean_cuantiles.FC_with_NA[, c("avg_cuantiles.FC_with_NA")])
  
  # Mean with not NA values scaled FC
  Mean_scaled.FC_with_NA <-  data %>%
    group_by(hour) %>%
    summarise(avg_scaled.FC_with_NA = mean(scaled.FC, na.rm = T)) %>%
    mutate(valor_orden = factor(hour, levels = valores_unicos)) %>%
    arrange(valor_orden)
  Mean_scaled.FC_with_NA <-
    data.frame(Mean_scaled.FC_with_NA[, c("avg_scaled.FC_with_NA")])
  
  # Merging all the data frames
  merged_df = data.frame(
    cbind(
      count_values,
      Missing_FC,
      Missing_SatO2,
      Mean_FC,
      Mean_SatO2,
      Mean_cuantiles.FC,
      Mean_scaled.FC,
      Mean_FC_with_NA,
      Mean_SatO2_with_NA,
      Mean_cuantiles.FC_with_NA,
      Mean_scaled.FC_with_NA
    )
  )
  
  colnames(merged_df) <-
    c(
      "hour",
      "n",
      "Missing_FC",
      "Missing_SatO2",
      "Mean_FC",
      "Mean_SatO2",
      "Mean_Q",
      "Mean_SC",
      "Mean_FC_NaN",
      "Mean_SatO2_NaN",
      "Mean_Q_NaN",
      "Mean_SC_NaN"
    )
  
  assign(paste0(name_variable, "_info"), merged_df)
  paste0("../data/info-patients/", name_variable, "_info", ".xlsx")
  # For written the tables in excel files
  write_xlsx(
    merged_df,
    paste0(
      "../data/info-patients/info/",
      name_variable,
      "_info",
      ".xlsx"
    )
  )
}
\end{lstlisting}


\subsubsection{Creación de Dataframes Individuales de \textit{Frecuencia Cardíaca} y de \textit{Saturación de Oxígeno} }\label{sec:anexo2-dataframes}

\begin{lstlisting}[style=mystyle,caption={Creación de Dataframes Individuales de \textit{Frecuencia Cardíaca} y de \textit{Saturación de Oxígeno}}, label={lst:dataframes-idividuales}]
    # Lets put together all the patients in the same data frame
M <- 1441
N <- length(file_patient_name)
FC_all_patients  <- as.data.frame(x = matrix(data = NA,
                                             nrow = M,
                                             ncol = N))
SatO2_all_patients  <- as.data.frame(x = matrix(data = NA,
                                                nrow = M,
                                                ncol = N))
colnames(FC_all_patients) = file_patient_name
colnames(SatO2_all_patients) = file_patient_name


## Imputing the data inside the created data frames
for (name_file in file_patient_name) {
  data = get(name_file)
  FC_all_patients[, paste0(name_file)] <- data$FC
  SatO2_all_patients[, paste0(name_file)] <- data$SatO2
}

#Adding an extra column for the time series reference
FC_all_patients$time <- c(1:M)
SatO2_all_patients$time <- c(1:M)


write_xlsx(FC_all_patients, "../data/FC_&_SatO2/FC_all_patients.xlsx")
write_xlsx(SatO2_all_patients,
           "../data/FC_&_SatO2/SatO2_all_patients.xlsx")
\end{lstlisting}
