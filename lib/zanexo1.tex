\section{ANEXO 1}\label{sec:anexo1}


\subsection{Código Input Gower}\label{sec:codigo-input-gower}

\vspace{-5pt}

\begin{multicols}{2} % Adjust the number of columns as desired
  \begin{lstlisting}[style=mystyle]
    Input_Gower <- function(dat, new_with_null, n_primeros) {
      # Juntamos los dos data_frames
      datos = rbind(dat, new_with_null)
      
      # Columnas del paciente introducido donde existen valores faltantes
      na_por_columna <- colSums(is.na(datos))
      
      if (sum(na_por_columna) == 0) {
        return(new_with_null)
      }
      
      # Columnas del paciente introducido con valores faltantes
      columnas_con_na <- names(which(na_por_columna > 0))
      selected_columns <- setdiff(names(datos), columnas_con_na)
      
      # Datos que se van a utilizar para calcular la distancia, se excluyen las
      # columnas que tengan valores faltantes
      datos_filtrados <- datos[, selected_columns]
      
      ## Se calcula la Distancia en el Total ##
      distance = gower.dist(data.x = datos_filtrados[-nrow(datos_filtrados),], data.y =  datos_filtrados[nrow(datos_filtrados),])
      
      # La salida es un data.frame, se convierte en array
      distance_array = array(distance)
      
      ######### Order #######
      distance_array_ordenado <- sort(distance_array)
      # Se ordena de menor a mayor
      n_primeros_pat <- head(distance_array_ordenado, n = n_primeros)
      # Se aislan las primeras n_primeros distancias mas pequeñas
      indices <- which(distance_array %in% n_primeros_pat)
      # Índices de los pacientes más cercanos
      
      datos[c(indices),]
      # Estos son los n_primeros datos más cercanos que se van a usar para imputar los demas
      
      # IMPUTAR DATOS #
      names_df1 <- names(datos)
      names_df2 <- names(datos_filtrados)
      
      # Se estudia las variables que no se compartem pues se imputaran datos en ellas
      columnas_no_compartidas <- setdiff(names_df1, names_df2)
      
      columnas_no_compartidas_numericas <-
        setdiff(names(datos[, sapply(datos, is.numeric)]),
                names(datos_filtrados[sapply(datos_filtrados, is.numeric),]))
    
      columnas_no_compartidas_character <-
        setdiff (columnas_no_compartidas,
                 columnas_no_compartidas_numericas)
      
      for (i in columnas_no_compartidas_numericas) {
        new_with_null[, i] = mean(datos[c(indices), i])
      }
      for (i in columnas_no_compartidas_character) {
        new_with_null[, i] <-
          names(table(datos[c(indices), i]))[which.max(table(datos[c(indices), i]))]
      }
      return(new_with_null)
    }
  \end{lstlisting}
\end{multicols}

\begin{longtable}{2} % Adjust the number of columns as desired
  \begin{lstlisting}[style=mystyle]
    Input_Gower <- function(dat, new_with_null, n_primeros) {
      # Juntamos los dos data_frames
      datos = rbind(dat, new_with_null)
      
      # Columnas del paciente introducido donde existen valores faltantes
      na_por_columna <- colSums(is.na(datos))
      
      if (sum(na_por_columna) == 0) {
        return(new_with_null)
      }
      
      # Columnas del paciente introducido con valores faltantes
      columnas_con_na <- names(which(na_por_columna > 0))
      selected_columns <- setdiff(names(datos), columnas_con_na)
      
      # Datos que se van a utilizar para calcular la distancia, se excluyen las
      # columnas que tengan valores faltantes
      datos_filtrados <- datos[, selected_columns]
      
      ## Se calcula la Distancia en el Total ##
      distance = gower.dist(data.x = datos_filtrados[-nrow(datos_filtrados),], data.y =  datos_filtrados[nrow(datos_filtrados),])
      
      # La salida es un data.frame, se convierte en array
      distance_array = array(distance)
      
      ######### Order #######
      distance_array_ordenado <- sort(distance_array)
      # Se ordena de menor a mayor
      n_primeros_pat <- head(distance_array_ordenado, n = n_primeros)
      # Se aislan las primeras n_primeros distancias mas pequeñas
      indices <- which(distance_array %in% n_primeros_pat)
      # Índices de los pacientes más cercanos
      
      datos[c(indices),]
      # Estos son los n_primeros datos más cercanos que se van a usar para imputar los demas
      
      # IMPUTAR DATOS #
      names_df1 <- names(datos)
      names_df2 <- names(datos_filtrados)
      
      # Se estudia las variables que no se compartem pues se imputaran datos en ellas
      columnas_no_compartidas <- setdiff(names_df1, names_df2)
      
      columnas_no_compartidas_numericas <-
        setdiff(names(datos[, sapply(datos, is.numeric)]),
                names(datos_filtrados[sapply(datos_filtrados, is.numeric),]))
    
      columnas_no_compartidas_character <-
        setdiff (columnas_no_compartidas,
                 columnas_no_compartidas_numericas)
      
      for (i in columnas_no_compartidas_numericas) {
        new_with_null[, i] = mean(datos[c(indices), i])
      }
      for (i in columnas_no_compartidas_character) {
        new_with_null[, i] <-
          names(table(datos[c(indices), i]))[which.max(table(datos[c(indices), i]))]
      }
      return(new_with_null)
    }
  \end{lstlisting}
\end{longtable}

