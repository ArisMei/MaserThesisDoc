\subsection{PRIMER ANEXO}\label{sec:anexo1}

\subsubsection{Visualización de los datos}\label{sec:codigo-visualizacion}
\lstset{style=mystyle2} 
\begin{lstlisting}[style=mystyle2,caption={Código Visualización de los Datos}, label={lst:codigo-visualizacion}]
  ---
  title: "Interactive Graphs"
  author: "Gonzalo Aris 16021"
  date: "2023-02-11"
  output:
    html_document:
      code_folding: hide
      highlight: textmate
      toc: yes
      toc_float:
        collapsed: no
      theme: cosmo
  runtime: shiny
  resource_files:
  - "data/raw-data/ACR-11231843.xlsx"
  - data/gather_FC_all_patients.xlsx
  ---
  
  ```{r setup, include=FALSE}
  knitr::opts_chunk$set(echo = TRUE)
  ```
  
  ```{r, warning=FALSE, echo=FALSE, message=FALSE}
  ### Libraries
  library(ggplot2)
  library(readxl)
  library(dplyr)
  library(writexl)
  library(data.table)
  library(readxl)
  library(shiny)
  library(readr)
  library(astsa)
  library(xlsx)
  library(ggpubr) # ggarrange
  ```
  
  ## Graphic Information
  
  ### Heart Rate and O2Sat
  
  ```{r, echo=FALSE, message=FALSE}
  #setwd("C:/Users/User/PROYECTOS/TFM-Gonzalo-Aris/deploying")
  gather_FC_all_patients <-
    read_excel("data/gather_FC_all_patients.xlsx")
  gather_SatO2_all_patients <-
    read_excel("data/gather_SatO2_all_patients.xlsx")
  file_patient_name <- read.csv("data/file_patient_name.csv")
  file_patient_name <- file_patient_name$x 
  ```
  
  ```{r}
  selectInput('patient',
              label = 'Patient: ',
              choices = file_patient_name,
              selected = "GLR_11225596")
  
  renderText(paste0("Patient: ", as.character(input$patient)))
  
  renderPlot({
    par(mar = c(4, 4, .1, .5))
    ggplot(gather_FC_all_patients[gather_FC_all_patients$time_series == as.character(input$patient), ]) +
      aes(x =  time, y = value, color = time_series) +
      geom_line(color = "black") +
      geom_point(color = "black") +
      theme(axis.text.x = element_text(angle = 60, hjust = 1)) + labs(
        title  = "Frecuencia Cardiaca",
        x = "Time",
        y = "BPM",
        subtitle = paste0(as.character(input$patient))
      )
  })
  
  renderPlot({
    #par(mar = c(4, 4, .1, .5))
    ggplot(gather_SatO2_all_patients[gather_SatO2_all_patients$time_series == as.character(input$patient), ]) +
      aes(x =  time, y = value, color = time_series) +
      geom_line(color = "black") +
      geom_point(color = "black") +
      
      theme(axis.text.x = element_text(angle = 60, hjust = 1)) + labs(
        title  = "Sat O2",
        x = "Time",
        y = "% O2",
        subtitle = paste0(as.character(input$patient))
      )
    
  })
  ```
  
  ## Imputed Data
  
  > Data imputation based on: <https://rpubs.com/garisj98/TimeSeriesImputation>
  
  ```{r, message=FALSE, warning=FALSE}
  file_patient_name_NO_DETERIORO <-
    data.frame(read_xlsx("data/file_patient_name_NO_DETERIORO.xlsx"))
  file_patient_name_NO_DETERIORO <- file_patient_name_NO_DETERIORO$x
  file_patient_name_DETERIORO <-
    data.frame(read_xlsx("data/file_patient_name_DETERIORO.xlsx"))
  file_patient_name_DETERIORO <- file_patient_name_DETERIORO$x
  
  
  #Imputed Data
  graph_data_FC = data.frame(read_xlsx("data/graph_data_FC.xlsx", col_names = TRUE))
  graph_data_SatO2 = data.frame(read_xlsx("data/graph_data_SatO2.xlsx", col_names = TRUE))
  ```
  
  ### Heart Rate
  
  ```{r}
  renderPlot({
    #par(mar = c(4, 4, .1, .5))
    ggplot(graph_data_FC, aes(x = time, y = graph_data_FC[, as.character(input$patient)])) +
      
      geom_line(color = "black") + xlab("") +
      
      geom_point(color = ifelse(graph_data_FC[, paste0(as.character(input$patient), ".1")] == TRUE, '#69b3a2', 'black')) +
      
      theme(axis.text.x = element_text(angle = 60, hjust = 1)) +
      
      labs(title  = "Heart Rate", subtitle = paste0(as.character(input$patient)))
    
  })
  ```
  
  ### SatO2
  
  ```{r}
  renderPlot({
    #par(mar = c(4, 4, .1, .5))
    ggplot(graph_data_SatO2, aes(x = time, y = graph_data_SatO2[, as.character(input$patient)])) +
      
      geom_line(color = "black") + xlab("") +
      
      geom_point(color = ifelse(graph_data_SatO2[, paste0(as.character(input$patient), ".1")] == TRUE, '#69b3a2', 'black')) +
      
      theme(axis.text.x = element_text(angle = 60, hjust = 1)) +
      
      labs(title  = "Sat O2", subtitle = paste0(as.character(input$patient)))
    
  })
  ```
  
  
  ## Time Series interface
  
  ```{r, message=FALSE, warning=FALSE}
  file_patient_name_NO_DETERIORO <-
    data.frame(read_xlsx("data/file_patient_name_NO_DETERIORO.xlsx"))
  file_patient_name_NO_DETERIORO <- file_patient_name_NO_DETERIORO$x
  file_patient_name_DETERIORO <-
    data.frame(read_xlsx("data/file_patient_name_DETERIORO.xlsx"))
  file_patient_name_DETERIORO <- file_patient_name_DETERIORO$x
  
  
  #Imputed Data
  FC_all_valid_patients_input = data.frame(
    read_xlsx(
      "data/FC_all_valid_patients_input.xlsx",
      sheet = "FC_all_valid_patients_input",
      col_names = TRUE
    )
  )
  
  FC_all_valid_patients_Binary_Mask_NA = data.frame(
    read_xlsx(
      "data/FC_all_valid_patients_input.xlsx",
      sheet = "Binary_Mask_NA",
      col_names = TRUE
    )
  )
  
  SatO2_all_valid_patients_input = data.frame(
    read_xlsx(
      "data/SatO2_all_valid_patients_input.xlsx",
      sheet = "SatO2_all_valid_patients_input",
      col_names = TRUE
    )
  )
  
  SatO2_all_patients_Binary_Mask_NA = data.frame(
    read_xlsx(
      "data/SatO2_all_valid_patients_input.xlsx",
      sheet = "Binary_Mask_NA",
      col_names = TRUE
    )
  )
  
  time = c(1:dim(FC_all_valid_patients_input)[1])
  ```
  
  ```{r}
  
  
  radioButtons(
    "select",
    label = "Select Deterioro or Not Deterioro",
    choices = list("DETERIORO" = 1, "NO DETERIORO" = 2),
    selected = 1
  )
  
  renderUI(if (input$select == 1) {
    selectInput(
      'patient_0',
      label = 'Patient:',
      choices = file_patient_name_DETERIORO,
      selected = "APA_11204819"
    )
  } else {
    selectInput(
      'patient_0',
      label = 'Patient:',
      choices = file_patient_name_NO_DETERIORO,
      selected = "ADAO_11159808"
    )
  })
  
  ```
  
  ### ACC Heart Rate and O2Sat
  
  ```{r}
  sliderInput(
    "slider_lag_1",
    label = "Lag Max ACC",
    min = 2,
    max = 200,
    value = 50
  )
  ```
  
  ```{r}
  renderPlot({
    acf(FC_all_valid_patients_input[, as.character(input$patient_0)],
        lag.max = input$slider_lag_1,
        main = "Auto- and Cross-Correlation Hear Rate")
  })
  renderPlot({
    acf(
      SatO2_all_valid_patients_input[, as.character(input$patient_0)],
      lag.max = input$slider_lag_1,
      main = "Auto- and Cross-Correlation SatO2"
    )
  })
  ```
  
  ### CCF between Heart Rate and O2Sat
  
  ```{r}
  sliderInput(
    "slider_lag_2",
    label = "Lag Max CCF",
    min = 2,
    max = 200,
    value = 50
  )
  ```
  
  ```{r}
  renderPlot({
    ccf(
      FC_all_valid_patients_input[, as.character(input$patient_0)],
      SatO2_all_valid_patients_input[, as.character(input$patient_0)],
      lag.max = input$slider_lag_2,
      ylab = 'CCF',
      main = "Cross Correlation"
    )
  })
  ```
  
\end{lstlisting}

\lstset{style=mystyle} 

\subsubsection{Inputación con Distancia de Gower}\label{sec:codigo-input-gower}

\paragraph{Función Imputación por el Método de Gower: Input\_Gower}\label{sec:codigo-input-gower-fun}


\begin{lstlisting}[style=mystyle,caption={Input\_Gower.R}, label={lst:input-gower-fun}]
  Input_Gower <- function(dat, new_with_null, n_primeros) {
  #### SELECCIÓN DE DATOS PARA IMPUTACIÓN ###
  # Juntamos los dos data.frames. Aquel que tiene todos los pacientes sin valores faltantes: `dat` y el data.frame que contiene al paciente con valores faltantes: `new_with_null`. Se añadirá a la última fila el paciente al que se le quieren imputar datos.
  datos = rbind(dat, new_with_null)
  
  # Columnas del paciente introducido donde existen valores faltantes
  na_por_columna <- colSums(is.na(datos))
  
  # Si no hay necesidad de imputar datos se devuelve directamente al paciente y se termina el proceso.
  if (sum(na_por_columna) == 0) {
    return(new_with_null)
  }
  
  columnas_con_na <-
    names(which(na_por_columna > 0)) # Nombres de las columnas del paciente introducido con valores faltantes
  selected_columns <-
    setdiff(names(datos), columnas_con_na) # Nombres de las columnas utilizadas para calcular la dist.Gower
  datos_filtrados <-
    datos[, selected_columns] # Datos que se van a utilizar para calcular la distancia, se excluyen las columnas que tengan valores faltantes
  
  #### CÁLCULO DIST.GOWER ###
  ## Se calcula la Distancia ##
  ### En data.y se encuentra el último paciente que es al que se le van a imputar los datos
  ### En data.x todos los demás
  distance = gower.dist(data.x = datos_filtrados[-nrow(datos_filtrados), ], data.y =  datos_filtrados[nrow(datos_filtrados), ])
  distance_array = array(distance)  # La salida es un data.frame, se convierte en array
  
  ### SELECCIÓN  DE PACIENTES ###
  distance_array_ordenado <-
    sort(distance_array) # Se ordenan las distancias de menor a mayor
  n_primeros_pat <-
    head(distance_array_ordenado, n = n_primeros) # Se aíslan las primeras n_primeros distancias mas pequeñas
  indices <-
    which(distance_array %in% n_primeros_pat) # Índices de los n_primeros pacientes más cercanos
  
  datos[c(indices), ] # Estos son los n_primeros datos más cercanos que se van a usar para imputar los demás
  
  ### IMPUTAR DATOS ###
  names_df1 <-
    names(datos) # Todas las variables, entre ellas las que se necesitan imputar
  names_df2 <-
    names(datos_filtrados) # Todas las variables que no se necesitan imputar
  columnas_no_compartidas <-
    setdiff(names_df1, names_df2) # Se estudia las variables que no se comparten pues se imputaran datos en ellas
  
  ### Se estudian por separado las variables Cuantitativas de las Cualitativas
  #### Cuantitativas:
  columnas_no_compartidas_numericas <-
    setdiff(names(datos[, sapply(datos, is.numeric)]),
            names(datos_filtrados[sapply(datos_filtrados, is.numeric), ]))
  
  #### Cualitativas:
  columnas_no_compartidas_character <-
    setdiff (columnas_no_compartidas,
             columnas_no_compartidas_numericas)
  
  #### Inputación Cuantitativas con la media de los n_primeros más cercanos
  for (i in columnas_no_compartidas_numericas) {
    print(i)
    new_with_null[, i] = mean(datos[c(indices), i])
    print(mean(datos[c(indices), i]))
  }
  
  #### Inputación Cualitativas con la mediana de los n_primeros más cercanos
  for (i in columnas_no_compartidas_character) {
    print(i)
    print(names(table(datos[c(indices), i]))[which.max(table(datos[c(indices), i]))])
    new_with_null[, i] <-
      names(table(datos[c(indices), i]))[which.max(table(datos[c(indices), i]))]
  }
  
  
  return(new_with_null) # Se devuelve al paciente con los datos imputados, es una "variable fila"}
\end{lstlisting}


\newpage

\paragraph{Función de preprocesamiento: data\_imputation\_Gower}\label{sec:codigo-input-gower-fun-preprocesamiento}

\begin{lstlisting}[style=mystyle,caption={data\_imputation\_Gower.R}, label={lst:input-gower-fun-preprocesamiento}]
  data_imputation_Gower <- function(df_merge_imputation, n_primeros) {
  # Pacientes con valores faltantes
  df_with_na <-
    df_merge_imputation[!complete.cases(df_merge_imputation),]
  # Pacientes que se van a utilizar para imputar los valores faltantes
  df_without_na <-
    df_merge_imputation[complete.cases(df_merge_imputation),]
  
  for (i in c(1:dim(df_with_na)[1])) {
    df_merge_imputation[row.names(df_with_na[i, ]), ] = Input_Gower(df_without_na, df_with_na[i, ], n_primeros)
  }
  
  return(df_merge_imputation)
}
\end{lstlisting}

\vspace{-5pt}
