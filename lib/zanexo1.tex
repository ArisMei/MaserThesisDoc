\subsection{Primer anexo} \label{sec:anexo1}

\subsubsection{Inputación con Distancia de Gower}\label{sec:codigo-input-gower}

\paragraph*{Función Imputación por el Método de Gower: Input\_Gower}\label{sec:codigo-input-gower-fun}


\begin{lstlisting}[style=mystyle,caption={Input\_Gower.R}, label={lst:input-gower-fun}]
  Input_Gower <- function(dat, new_with_null, n_primeros) {
  #### SELECCIÓN DE DATOS PARA IMPUTACIÓN ###
  # Juntamos los dos data.frames. Aquel que tiene todos los pacientes sin valores faltantes: `dat` y el data.frame que contiene al paciente con valores faltantes: `new_with_null`. Se añadirá a la última fila el paciente al que se le quieren imputar datos.
  datos = rbind(dat, new_with_null)
  
  # Columnas del paciente introducido donde existen valores faltantes
  na_por_columna <- colSums(is.na(datos))
  
  # Si no hay necesidad de imputar datos se devuelve directamente al paciente y se termina el proceso.
  if (sum(na_por_columna) == 0) {
    return(new_with_null)
  }
  
  columnas_con_na <-
    names(which(na_por_columna > 0)) # Nombres de las columnas del paciente introducido con valores faltantes
  selected_columns <-
    setdiff(names(datos), columnas_con_na) # Nombres de las columnas utilizadas para calcular la dist.Gower
  datos_filtrados <-
    datos[, selected_columns] # Datos que se van a utilizar para calcular la distancia, se excluyen las columnas que tengan valores faltantes
  
  #### CÁLCULO DIST.GOWER ###
  ## Se calcula la Distancia ##
  ### En data.y se encuentra el último paciente que es al que se le van a imputar los datos
  ### En data.x todos los demás
  distance = gower.dist(data.x = datos_filtrados[-nrow(datos_filtrados), ], data.y =  datos_filtrados[nrow(datos_filtrados), ])
  distance_array = array(distance)  # La salida es un data.frame, se convierte en array
  
  ### SELECCIÓN  DE PACIENTES ###
  distance_array_ordenado <-
    sort(distance_array) # Se ordenan las distancias de menor a mayor
  n_primeros_pat <-
    head(distance_array_ordenado, n = n_primeros) # Se aíslan las primeras n_primeros distancias mas pequeñas
  indices <-
    which(distance_array %in% n_primeros_pat) # Índices de los n_primeros pacientes más cercanos
  
  datos[c(indices), ] # Estos son los n_primeros datos más cercanos que se van a usar para imputar los demás
  
  ### IMPUTAR DATOS ###
  names_df1 <-
    names(datos) # Todas las variables, entre ellas las que se necesitan imputar
  names_df2 <-
    names(datos_filtrados) # Todas las variables que no se necesitan imputar
  columnas_no_compartidas <-
    setdiff(names_df1, names_df2) # Se estudia las variables que no se comparten pues se imputaran datos en ellas
  
  ### Se estudian por separado las variables Cuantitativas de las Cualitativas
  #### Cuantitativas:
  columnas_no_compartidas_numericas <-
    setdiff(names(datos[, sapply(datos, is.numeric)]),
            names(datos_filtrados[sapply(datos_filtrados, is.numeric), ]))
  
  #### Cualitativas:
  columnas_no_compartidas_character <-
    setdiff (columnas_no_compartidas,
             columnas_no_compartidas_numericas)
  
  #### Inputación Cuantitativas con la media de los n_primeros más cercanos
  for (i in columnas_no_compartidas_numericas) {
    print(i)
    new_with_null[, i] = mean(datos[c(indices), i])
    print(mean(datos[c(indices), i]))
  }
  
  #### Inputación Cualitativas con la mediana de los n_primeros más cercanos
  for (i in columnas_no_compartidas_character) {
    print(i)
    print(names(table(datos[c(indices), i]))[which.max(table(datos[c(indices), i]))])
    new_with_null[, i] <-
      names(table(datos[c(indices), i]))[which.max(table(datos[c(indices), i]))]
  }
  
  
  return(new_with_null) # Se devuelve al paciente con los datos imputados, es una "variable fila"
}
\end{lstlisting}


\newpage

\paragraph*{Función de preprocesamiento: data\_imputation\_Gower}\label{sec:codigo-input-gower-fun-preprocesamiento}

\begin{lstlisting}[style=mystyle,caption={data\_imputation\_Gower.R}, label={lst:input-gower-fun-preprocesamiento}]
  data_imputation_Gower <- function(df_merge_imputation, n_primeros) {
  # Pacientes con valores faltantes
  df_with_na <-
    df_merge_imputation[!complete.cases(df_merge_imputation),]
  # Pacientes que se van a utilizar para imputar los valores faltantes
  df_without_na <-
    df_merge_imputation[complete.cases(df_merge_imputation),]
  
  for (i in c(1:dim(df_with_na)[1])) {
    df_merge_imputation[row.names(df_with_na[i, ]), ] = Input_Gower(df_without_na, df_with_na[i, ], n_primeros)
  }
  
  return(df_merge_imputation)
}
\end{lstlisting}

\vspace{-5pt}
