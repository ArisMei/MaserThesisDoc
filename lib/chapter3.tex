\section{METHODOLOGY}\label{sec:methodology}

In this chapter, we describe the methodology we used to achieve our research objectives. We begin by discussing the research design and data collection methods we employed to gather information on heart rate variability in infants with heart disease. We then outline our approach to data preprocessing, including data cleaning and feature selection. Finally, we describe the machine learning algorithms we used to classify the temporal series data and the performance metrics we used to evaluate the models. Depending on the specifics of your research, you may want to include a section on data collection and preprocessing, as well as any ethical considerations you had to take into account.


\subsection{Research design}

We employed a quantitative research design to investigate the heart rate variability in infants with heart disease. Our study involved the analysis of temporal series data collected from electrocardiogram (ECG) recordings of infants with heart disease.

\subsection{Data Collection}

We collected the data for our study from a pediatric hospital, where we recruited infants diagnosed with heart disease. We obtained informed consent from the parents or guardians of the infants before recording the ECG data. The ECG recordings were taken over a period of 24 hours, and we collected a total of X hours of data.

\subsection{Data Preprocessing}

We preprocessed the ECG data using a custom pipeline, which included data cleaning and feature selection. We removed any noisy or missing data from the ECG recordings, and we extracted a set of relevant features that we believed were important for classification of heart rate variability.

\subsection{Machine Learning Algorithms}

We employed several machine learning algorithms to classify the heart rate variability in infants with heart disease. These included Random Forest, Support Vector Machines, and Long Short-Term Memory networks. We trained each model on a subset of the data, and we used cross-validation to assess their performance.

\subsection{Performance Metrics}

We evaluated the performance of each machine learning model using several metrics, including accuracy, precision, recall, and F1 score. We also used confusion matrices and Receiver Operating Characteristic (ROC) curves to assess the models' ability to correctly classify the ECG data.

By following this methodology, we were able to achieve our research objectives and gain insights into the heart rate variability in infants with heart disease. Our results and analysis are presented in Chapter 5.


